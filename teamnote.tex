% Team Note of Deobureo Minkyu Party.
% These codes should be guaranteed, fast enough, short and easy to type.

\documentclass[landscape, 8pt, a4paper, oneside, twocolumn]{extarticle}

\usepackage[compact]{titlesec}
\titlespacing*{\section}
{0pt}{0px plus 1px minus 0px}{-2px plus 0px minus 0px}
\titlespacing*{\subsection}
{0pt}{0px plus 1px minus 0px}{0px plus 3px minus 3px}

\setlength{\columnseprule}{0.4pt}
\pagenumbering{arabic}

\usepackage{kotex}

\usepackage[left=0.8cm, right=0.8cm, top=2cm, bottom=0.3cm, a4paper]{geometry}
\usepackage{amsmath}
\usepackage{ulem}
\usepackage{amssymb}
\usepackage{minted}
\usepackage{color, hyperref}
\usepackage{indentfirst}
\usepackage{enumitem}

\usepackage{fancyhdr}
\usepackage{lastpage}
\pagestyle{fancy}
\lhead{KAIST - Deobureo Minkyu Party.}
\rhead{Page \thepage  \ of \pageref{LastPage} }
\fancyfoot{}

\headsep 0.2cm

\setminted{breaklines=true, tabsize=2, breaksymbolleft=}
\usemintedstyle{perldoc}

\setlength\partopsep{-\topsep -\parskip}

\title{Team Note of Deobureo Minkyu Party}
\author{tncks0121, koosaga, alex9801, hyea (alumni)}
\date{Compiled on \today}

\newcommand{\revised}{Should be \textcolor{red}{\textbf{revised}}.}
\newcommand{\tested}{Should be \textcolor{red}{\textbf{tested}}.}
\newcommand{\added}{Should be \textcolor{red}{\textbf{added}}.}
\newcommand{\WIP}{\textcolor{red}{\textbf{WIP}}}
\begin{document}

{
	\Large

	\maketitle

\tableofcontents
}
\thispagestyle{fancy}
\pagebreak

\textcolor{red}{\textbf{ALL BELOW HERE ARE USELESS IF YOU READ THE STATEMENT WRONG}}

\section{Flows, Matching}
\subsection {Hopcroft-Karp Bipartite Matching}
\begin{minted}{cpp}
const int MAXN = 50005, MAXM = 50005;
vector<int> gph[MAXN];
int dis[MAXN], l[MAXN], r[MAXM], vis[MAXN];
void clear(){ for(int i=0; i<MAXN; i++) gph[i].clear();	}
void add_edge(int l, int r){ gph[l].push_back(r); }
bool bfs(int n){
	queue<int> que;
	bool ok = 0;
	memset(dis, 0, sizeof(dis));
	for(int i=0; i<n; i++){
		if(l[i] == -1 && !dis[i]){
			que.push(i);
			dis[i] = 1;
		}
	}
	while(!que.empty()){
		int x = que.front();
		que.pop();
		for(auto &i : gph[x]){
			if(r[i] == -1) ok = 1;
			else if(!dis[r[i]]){
				dis[r[i]] = dis[x] + 1;
				que.push(r[i]);
			}
		}
	}
	return ok;
}
bool dfs(int x){
	if(vis[x]) return 0;
	vis[x] = 1;
	for(auto &i : gph[x]){
		if(r[i] == -1 || (!vis[r[i]] && dis[r[i]] == dis[x] + 1 && dfs(r[i]))){
			l[x] = i; r[i] = x;
			return 1;
		}
	}
	return 0;
}
int match(int n){
	memset(l, -1, sizeof(l));
	memset(r, -1, sizeof(r));
	int ret = 0;
	while(bfs(n)){
		memset(vis, 0, sizeof(vis));
		for(int i=0; i<n; i++) if(l[i] == -1 && dfs(i)) ret++;
	}
	return ret;
}
bool chk[MAXN + MAXM];
void rdfs(int x, int n){
	if(chk[x]) return;
	chk[x] = 1;
	for(auto &i : gph[x]){
		chk[i + n] = 1;
		rdfs(r[i], n);
	}
}
vector<int> getcover(int n, int m){ // solve min. vertex cover
	match(n);
	memset(chk, 0, sizeof(chk));
	for(int i=0; i<n; i++) if(l[i] == -1) rdfs(i, n);
	vector<int> v;
	for(int i=0; i<n; i++) if(!chk[i]) v.push_back(i);
	for(int i=n; i<n+m; i++) if(chk[i]) v.push_back(i);
	return v;
}
\end{minted}
\subsection {Dinic's Algorithm}
\begin{minted}{cpp}
const int MAXN = 505;
struct edg{ int pos, cap, rev; };
vector<edg> gph[MAXN];
void clear(){ for(int i=0; i<MAXN; i++) gph[i].clear(); }
void add_edge(int s, int e, int x){
	gph[s].push_back({e, x, (int)gph[e].size()});
	gph[e].push_back({s, 0, (int)gph[s].size()-1});
}
int dis[MAXN], pnt[MAXN];
bool bfs(int src, int sink){
	memset(dis, 0, sizeof(dis));
	memset(pnt, 0, sizeof(pnt));
	queue<int> que;
	que.push(src);
	dis[src] = 1;
	while(!que.empty()){
		int x = que.front();
		que.pop();
		for(auto &e : gph[x]){
			if(e.cap > 0 && !dis[e.pos]){
				dis[e.pos] = dis[x] + 1;
				que.push(e.pos);
			}
		}
	}
	return dis[sink] > 0;
}
int dfs(int x, int sink, int f){
	if(x == sink) return f;
	for(; pnt[x] < gph[x].size(); pnt[x]++){
		edg e = gph[x][pnt[x]];
		if(e.cap > 0 && dis[e.pos] == dis[x] + 1){
			int w = dfs(e.pos, sink, min(f, e.cap));
			if(w){
				gph[x][pnt[x]].cap -= w;
				gph[e.pos][e.rev].cap += w;
				return w;
			}
		}
	}
	return 0;
}
lint match(int src, int sink){
	lint ret = 0;
	while(bfs(src, sink)){
		int r;
		while((r = dfs(src, sink, 2e9))) ret += r;
	}
	return ret;
}
\end{minted}
\subsection {Min Cost Max Flow}
\begin{minted}{cpp}
const int MAXN = 100;
struct edg{ int pos, cap, rev, cost; };
vector<edg> gph[MAXN];
void clear(){
	for(int i=0; i<MAXN; i++) gph[i].clear();
}
void add_edge(int s, int e, int x, int c){
	gph[s].push_back({e, x, (int)gph[e].size(), c});
	gph[e].push_back({s, 0, (int)gph[s].size()-1, -c});
}
int dist[MAXN], pa[MAXN], pe[MAXN];
bool inque[MAXN];
bool spfa(int src, int sink){
	memset(dist, 0x3f, sizeof(dist));
	memset(inque, 0, sizeof(inque));
	queue<int> que;
	dist[src] = 0;
	inque[src] = 1;
	que.push(src);
	bool ok = 0;
	while(!que.empty()){
		int x = que.front();
		que.pop();
		if(x == sink) ok = 1;
		inque[x] = 0;
		for(int i=0; i<gph[x].size(); i++){
			edg e = gph[x][i];
			if(e.cap > 0 && dist[e.pos] > dist[x] + e.cost){
				dist[e.pos] = dist[x] + e.cost;
				pa[e.pos] = x;
				pe[e.pos] = i;
				if(!inque[e.pos]){
					inque[e.pos] = 1;
					que.push(e.pos);
				}
			}
		}
	}
	return ok;
}
int match(int src, int sink){
	int ret = 0;
	while(spfa(src, sink)){
		int cap = 1e9;
		for(int pos = sink; pos != src; pos = pa[pos]){
			cap = min(cap, gph[pa[pos]][pe[pos]].cap);
		}
		ret += dist[sink] * cap;
		for(int pos = sink; pos != src; pos = pa[pos]){
			int rev = gph[pa[pos]][pe[pos]].rev;
			gph[pa[pos]][pe[pos]].cap -= cap;
			gph[pos][rev].cap += cap;
		}
	}
	return ret;
}
\end{minted}
\subsection{Hell-Joseon style MCMF}
\begin{minted}{cpp}
const int MAXN = 100;
struct edg{ int pos, cap, rev, cost; };
vector<edg> gph[MAXN];
void clear(){ for(int i=0; i<MAXN; i++) gph[i].clear(); }
void add_edge(int s, int e, int x, int c){
	gph[s].push_back({e, x, (int)gph[e].size(), c});
	gph[e].push_back({s, 0, (int)gph[s].size()-1, -c});
}
int phi[MAXN], inque[MAXN], dist[MAXN];
void prep(int src, int sink){
	memset(phi, 0x3f, sizeof(phi));
	memset(dist, 0x3f, sizeof(dist));
	queue<int> que;
	que.push(src);
	inque[src] = 1;
	while(!que.empty()){
		int x = que.front();
		que.pop();
		inque[x] = 0;
		for(auto &i : gph[x]){
			if(i.cap > 0 && phi[i.pos] > phi[x] + i.cost){
				phi[i.pos] = phi[x] + i.cost;
				if(!inque[i.pos]){
					inque[i.pos] = 1;
					que.push(i.pos);
				}
			}
		}
	}
	for(int i=0; i<MAXN; i++){
		for(auto &j : gph[i]){
			if(j.cap > 0) j.cost += phi[i] - phi[j.pos];
		}
	}
	priority_queue<pi, vector<pi>, greater<pi> > pq;
	pq.push(pi(0, src));
	dist[src] = 0;
	while(!pq.empty()){
		auto l = pq.top();
		pq.pop();
		if(dist[l.second] != l.first) continue;
		for(auto &i : gph[l.second]){
			if(i.cap > 0 && dist[i.pos] > l.first + i.cost){
				dist[i.pos] = l.first + i.cost;
				pq.push(pi(dist[i.pos], i.pos));
			}
		}
	}
}
bool vis[MAXN];
int ptr[MAXN];
int dfs(int pos, int sink, int flow){
	vis[pos] = 1;
	if(pos == sink) return flow;
	for(; ptr[pos] < gph[pos].size(); ptr[pos]++){
		auto &i = gph[pos][ptr[pos]];
		if(!vis[i.pos] && dist[i.pos] == i.cost + dist[pos] && i.cap > 0){
			int ret = dfs(i.pos, sink, min(i.cap, flow));
			if(ret != 0){
				i.cap -= ret;
				gph[i.pos][i.rev].cap += ret;
				return ret;
			}
		}
	}
	return 0;
}
int match(int src, int sink, int sz){
	prep(src, sink);
	for(int i=0; i<sz; i++) dist[i] += phi[sink] - phi[src];
	int ret = 0;
	while(true){
		memset(ptr, 0, sizeof(ptr));
		memset(vis, 0, sizeof(vis));
		int tmp = 0;
		while((tmp = dfs(src, sink, 1e9))){
			ret += dist[sink] * tmp;
			memset(vis, 0, sizeof(vis));
		}
		tmp = 1e9;
		for(int i=0; i<sz; i++){
			if(!vis[i]) continue;
			for(auto &j : gph[i]){
				if(j.cap > 0 && !vis[j.pos]){
					tmp = min(tmp, (dist[i] + j.cost) - dist[j.pos]);
				}
			}
		}
		if(tmp > 1e9 - 200) break;
		for(int i=0; i<sz; i++){
			if(!vis[i]) dist[i] += tmp;
		}
	}
	return ret;
}
\end{minted}
\subsection {Circulation Problem}
\begin{minted}{cpp}
maxflow mf;
lint lsum;
void clear(){
	lsum = 0;
	mf.clear();
}
void add_edge(int s, int e, int l, int r){
	lsum += l;
	mf.add_edge(s + 2, e + 2, r - l);
	mf.add_edge(0, e + 2, l);
	mf.add_edge(s + 2, 1, l);
}
bool solve(int s, int e){
	mf.add_edge(e+2, s+2, 1e9); // to reduce as maxflow with lower bounds, in circulation problem skip this line
	return lsum == mf.match(0, 1);
	// to get maximum LR flow, run maxflow from s+2 to e+2 again
}
\end{minted}
\subsection {Min Cost Circulation}
\begin{minted}{cpp}
// Cycle canceling (Dual of successive shortest path)
// Time complexity is ridiculously high (F * maxC * nm^2). But runs reasonably in practice (V = 70 in 1s)
struct edg{ int pos, cap, rev, cost; };
vector<edg> gph[MAXN];
void clear(){ for(int i=0; i<MAXN; i++) gph[i].clear(); }
void add_edge(int s, int e, int x, int c){
	gph[s].push_back({e, x, (int)gph[e].size(), c});
	gph[e].push_back({s, 0, (int)gph[s].size()-1, -c});
}
int dist[MAXN], par[MAXN], pae[MAXN];
int negative_cycle(int n){
	bool mark[MAXN] = {};
	memset(dist, 0, sizeof(dist));
	int upd = -1;
	for(int i=0; i<=n; i++){
		for(int j=0; j<n; j++){
			int idx = 0;
			for(auto &k : gph[j]){
				if(k.cap > 0 && dist[k.pos] > dist[j] + k.cost){
					dist[k.pos] = dist[j] + k.cost;
					par[k.pos] = j;
					pae[k.pos] = idx;
					if(i == n){
						upd = j;
						while(!mark[upd]){
							mark[upd] = 1;
							upd = par[upd];
						}
						return upd;
					}
				}
				idx++;
			}
		}
	}
	return -1;
}
int match(int n){
	int rt = -1;
	int ans = 0;
	while(~(rt = negative_cycle(n))){
		bool mark[MAXN] = {};
		vector<pi> cyc;
		while(!mark[rt]){
			cyc.push_back(pi(par[rt], pae[rt]));
			mark[rt] = 1;
			rt = par[rt];
		}
		reverse(cyc.begin(), cyc.end());
		int capv = 1e9;
		for(auto &i : cyc){
			auto e = &gph[i.first][i.second];
			capv = min(capv, e->cap);
		}
		for(auto &i : cyc){
			auto e = &gph[i.first][i.second];
			e->cap -= capv;
			gph[e->pos][e->rev].cap += capv;
			ans += e->cost * capv;
		}
	}
	return ans;
}
\end{minted}
\subsection {Gomory-Hu Tree}
\begin{minted}{cpp}
struct edg{ int s, e, x; };
vector<edg> edgs;
maxflow mf;
void clear(){ edgs.clear(); }
void add_edge(int s, int e, int x){ edgs.push_back({s, e, x}); }
bool vis[MAXN];
void dfs(int x){
	if(vis[x]) return;
	vis[x] = 1;
	for(auto &i : mf.gph[x]) if(i.cap > 0) dfs(i.pos);
}
vector<pi> solve(int n){ // i - j cut : i - j minimum edge cost. 0 based.
	vector<pi> ret(n); // if i > 0, stores pair(parent,cost)
	for(int i=1; i<n; i++){
		for(auto &j : edgs){
			mf.add_edge(j.s, j.e, j.x);
			mf.add_edge(j.e, j.s, j.x);
		}
		ret[i].first = mf.match(i, ret[i].second);
		memset(vis, 0, sizeof(vis));
		dfs(i);
		for(int j=i+1; j<n; j++){
			if(ret[j].second == ret[i].second && vis[j]){
				ret[j].second = i;
			}
		}
		mf.clear();
	}
	return ret;
}
\end{minted}
\subsection{Blossom Algorithm for General Matching}
\begin{minted}{cpp}
const int MAXN = 2020 + 1;
// 1-based Vertex index
int vis[MAXN], par[MAXN], orig[MAXN], match[MAXN], aux[MAXN], t, N;
vector<int> conn[MAXN];
queue<int> Q;
void addEdge(int u, int v) {
	conn[u].push_back(v); conn[v].push_back(u);
}
void init(int n) {
	N = n; t = 0;
	for(int i=0; i<=n; ++i) {
		conn[i].clear();
		match[i] = aux[i] = par[i] = 0;
	}
}
void augment(int u, int v) {
	int pv = v, nv;
	do {
		pv = par[v]; nv = match[pv];
		match[v] = pv; match[pv] = v;
		v = nv;
	} while(u != pv);
}
int lca(int v, int w) {
	++t;
	while(true) {
		if(v) {
			if(aux[v] == t) return v; aux[v] = t;
			v = orig[par[match[v]]];
		}
		swap(v, w);
	}
}
void blossom(int v, int w, int a) {
	while(orig[v] != a) {
		par[v] = w; w = match[v];
		if(vis[w] == 1) Q.push(w), vis[w] = 0;
		orig[v] = orig[w] = a;
		v = par[w];
	}
}
bool bfs(int u) {
	fill(vis+1, vis+1+N, -1); iota(orig + 1, orig + N + 1, 1);
	Q = queue<int> (); Q.push(u); vis[u] = 0;
	while(!Q.empty()) {
		int v = Q.front(); Q.pop();
		for(int x: conn[v]) {
			if(vis[x] == -1) {
				par[x] = v; vis[x] = 1;
				if(!match[x]) return augment(u, x), true;
				Q.push(match[x]); vis[match[x]] = 0;
			}
			else if(vis[x] == 0 && orig[v] != orig[x]) {
				int a = lca(orig[v], orig[x]);
				blossom(x, v, a); blossom(v, x, a);
			}
		}
	}
	return false;
}
int Match() {
	int ans = 0;
	// find random matching (not necessary, constant improvement)
	vector<int> V(N-1); iota(V.begin(), V.end(), 1);
	shuffle(V.begin(), V.end(), mt19937(0x94949));
	for(auto x: V) if(!match[x]){
		for(auto y: conn[x]) if(!match[y]) {
			match[x] = y, match[y] = x;
			++ans; break;
		}
	}
	for(int i=1; i<=N; ++i) if(!match[i] && bfs(i)) ++ans;
	return ans;
}
\end{minted}
\subsection {Blossom Algorithm for Weighted General Matching}
\begin{minted}{cpp}
// N^3 (but fast in practice)
static const int INF = INT_MAX;
static const int N = 514;
struct edge{
	int u,v,w; edge(){}
	edge(int ui,int vi,int wi)
		:u(ui),v(vi),w(wi){}
};
int n,n_x;
edge g[N*2][N*2];
int lab[N*2];
int match[N*2],slack[N*2],st[N*2],pa[N*2];
int flo_from[N*2][N+1],S[N*2],vis[N*2];
vector<int> flo[N*2];
queue<int> q;
int e_delta(const edge &e){
	return lab[e.u]+lab[e.v]-g[e.u][e.v].w*2;
}
void update_slack(int u,int x){
	if(!slack[x]||e_delta(g[u][x])<e_delta(g[slack[x]][x]))slack[x]=u;
}
void set_slack(int x){
	slack[x]=0;
	for(int u=1;u<=n;++u)
		if(g[u][x].w>0&&st[u]!=x&&S[st[u]]==0)
			update_slack(u,x);
}
void q_push(int x){
	if(x<=n)q.push(x);
	else for(size_t i=0;i<flo[x].size();i++)
		q_push(flo[x][i]);
}
void set_st(int x,int b){
	st[x]=b;
	if(x>n)for(size_t i=0;i<flo[x].size();++i)
		set_st(flo[x][i],b);
}
int get_pr(int b,int xr){
	int pr=find(flo[b].begin(),flo[b].end(),xr)-flo[b].begin();
	if(pr%2==1){
		reverse(flo[b].begin()+1,flo[b].end());
		return (int)flo[b].size()-pr;
	}else return pr;
}
void set_match(int u,int v){
	match[u]=g[u][v].v;
	if(u<=n) return;
	edge e=g[u][v];
	int xr=flo_from[u][e.u],pr=get_pr(u,xr);
	for(int i=0;i<pr;++i)set_match(flo[u][i],flo[u][i^1]);
	set_match(xr,v);
	rotate(flo[u].begin(),flo[u].begin()+pr,flo[u].end());
}
void augment(int u,int v){
	for(;;){
		int xnv=st[match[u]];
		set_match(u,v);
		if(!xnv)return;
		set_match(xnv,st[pa[xnv]]);
		u=st[pa[xnv]],v=xnv;
	}
}
int get_lca(int u,int v){
	static int t=0;
	for(++t;u||v;swap(u,v)){
		if(u==0)continue;
		if(vis[u]==t)return u;
		vis[u]=t;
		u=st[match[u]];
		if(u)u=st[pa[u]];
	}
	return 0;
}
void add_blossom(int u,int lca,int v){
	int b=n+1;
	while(b<=n_x&&st[b])++b;
	if(b>n_x)++n_x;
	lab[b]=0,S[b]=0;
	match[b]=match[lca];
	flo[b].clear();
	flo[b].push_back(lca);
	for(int x=u,y;x!=lca;x=st[pa[y]])
		flo[b].push_back(x),flo[b].push_back(y=st[match[x]]),q_push(y);
	reverse(flo[b].begin()+1,flo[b].end());
	for(int x=v,y;x!=lca;x=st[pa[y]])
		flo[b].push_back(x),flo[b].push_back(y=st[match[x]]),q_push(y);
	set_st(b,b);
	for(int x=1;x<=n_x;++x)g[b][x].w=g[x][b].w=0;
	for(int x=1;x<=n;++x)flo_from[b][x]=0;
	for(size_t i=0;i<flo[b].size();++i){
		int xs=flo[b][i];
		for(int x=1;x<=n_x;++x)
			if(g[b][x].w==0||e_delta(g[xs][x])<e_delta(g[b][x]))
				g[b][x]=g[xs][x],g[x][b]=g[x][xs];
		for(int x=1;x<=n;++x)
			if(flo_from[xs][x])flo_from[b][x]=xs;
	}
	set_slack(b);
}
void expand_blossom(int b){
	for(size_t i=0;i<flo[b].size();++i)
		set_st(flo[b][i],flo[b][i]);
	int xr=flo_from[b][g[b][pa[b]].u],pr=get_pr(b,xr);
	for(int i=0;i<pr;i+=2){
		int xs=flo[b][i],xns=flo[b][i+1];
		pa[xs]=g[xns][xs].u;
		S[xs]=1,S[xns]=0;
		slack[xs]=0,set_slack(xns);
		q_push(xns);
	}
	S[xr]=1,pa[xr]=pa[b];
	for(size_t i=pr+1;i<flo[b].size();++i){
		int xs=flo[b][i];
		S[xs]=-1,set_slack(xs);
	}
	st[b]=0;
}
bool on_found_edge(const edge &e){
	int u=st[e.u],v=st[e.v];
	if(S[v]==-1){
		pa[v]=e.u,S[v]=1;
		int nu=st[match[v]];
		slack[v]=slack[nu]=0;
		S[nu]=0,q_push(nu);
	}else if(S[v]==0){
		int lca=get_lca(u,v);
		if(!lca)return augment(u,v),augment(v,u),true;
		else add_blossom(u,lca,v);
	}
	return false;
}
bool matching(){
	memset(S+1,-1,sizeof(int)*n_x);
	memset(slack+1,0,sizeof(int)*n_x);
	q=queue<int>();
	for(int x=1;x<=n_x;++x)
		if(st[x]==x&&!match[x])pa[x]=0,S[x]=0,q_push(x);
	if(q.empty())return false;
	for(;;){
		while(q.size()){
			int u=q.front();q.pop();
			if(S[st[u]]==1)continue;
			for(int v=1;v<=n;++v)
				if(g[u][v].w>0&&st[u]!=st[v]){
					if(e_delta(g[u][v])==0){
						if(on_found_edge(g[u][v]))return true;
					}else update_slack(u,st[v]);
				}
		}
		int d=INF;
		for(int b=n+1;b<=n_x;++b)
			if(st[b]==b&&S[b]==1)d=min(d,lab[b]/2);
		for(int x=1;x<=n_x;++x)
			if(st[x]==x&&slack[x]){
				if(S[x]==-1)d=min(d,e_delta(g[slack[x]][x]));
				else if(S[x]==0)d=min(d,e_delta(g[slack[x]][x])/2);
			}
		for(int u=1;u<=n;++u){
			if(S[st[u]]==0){
				if(lab[u]<=d)return 0;
				lab[u]-=d;
			}else if(S[st[u]]==1)lab[u]+=d;
		}
		for(int b=n+1;b<=n_x;++b)
			if(st[b]==b){
				if(S[st[b]]==0)lab[b]+=d*2;
				else if(S[st[b]]==1)lab[b]-=d*2;
			}
		q=queue<int>();
		for(int x=1;x<=n_x;++x)
			if(st[x]==x&&slack[x]&&st[slack[x]]!=x&&e_delta(g[slack[x]][x])==0)
				if(on_found_edge(g[slack[x]][x]))return true;
		for(int b=n+1;b<=n_x;++b)
			if(st[b]==b&&S[b]==1&&lab[b]==0)expand_blossom(b);
	}
	return false;
}
pair<long long,int> solve(){
	memset(match+1,0,sizeof(int)*n);
	n_x=n;
	int n_matches=0;
	long long tot_weight=0;
	for(int u=0;u<=n;++u)st[u]=u,flo[u].clear();
	int w_max=0;
	for(int u=1;u<=n;++u)
		for(int v=1;v<=n;++v){
			flo_from[u][v]=(u==v?u:0);
			w_max=max(w_max,g[u][v].w);
		}
	for(int u=1;u<=n;++u)lab[u]=w_max;
	while(matching())++n_matches;
	for(int u=1;u<=n;++u)
		if(match[u]&&match[u]<u)
			tot_weight+=g[u][match[u]].w;
	return make_pair(tot_weight,n_matches);
}
void add_edge( int ui , int vi , int wi ){
	g[ui][vi].w = g[vi][ui].w = wi;
}
void init( int _n ){
	n = _n;
	for(int u=1;u<=n;++u)
		for(int v=1;v<=n;++v)
			g[u][v]=edge(u,v,0);
}
\end{minted}
\section{Graph}
\subsection {2-SAT}
\begin{minted}{cpp}
strongly_connected scc;
int n; // = number of clauses
void init(int _n){ scc.clear(); n = _n; }
int NOT(int x){ return x >= n ? (x - n) : (x + n); }
void add_edge(int x, int y){ // input ~x to denote NOT
	if((x >> 31) & 1) x = (~x) + n;
	if((y >> 31) & 1) y = (~y) + n;
	scc.add_edge(x, y), scc.add_edge(NOT(y), NOT(x));
}
bool satisfy(vector<bool> &res){
	res.resize(n);
	scc.get_scc(2*n);
	for(int i=0; i<n; i++){
		if(scc.comp[i] == scc.comp[NOT(i)]) return 0;
		if(scc.comp[i] < scc.comp[NOT(i)]) res[i] = 0;
		else res[i] = 1;
	}
	return 1;
}
\end{minted}
\subsection {BCC}
\begin{minted}{cpp}
void color(int x, int p){
	if(p){
		bcc[p].push_back(x);
		cmp[x].push_back(p);
	}
	for(auto &i : gph[x]){
		if(cmp[i].size()) continue;
		if(low[i] >= dfn[x]){
			bcc[++c].push_back(x);
			cmp[x].push_back(c);
			color(i, c);
		}
		else color(i, p);
	}
}
\end{minted}
\subsection{Splay Tree + Link-Cut Tree}
\begin{minted}{cpp}
// Checklist 1. Is it link cut, or splay?
// Checklist 2. In link cut, is son always root?
void rotate(node *x){
	if(!x->p) return;
	push(x->p);  // if there's lazy stuff
	push(x);
	node *p = x->p;
	bool is_left = (p->l == x);
	node *b = (is_left ? x->r : x->l);
	x->p = p->p;
	if(x->p && x->p->l == p) x->p->l = x;
	if(x->p && x->p->r == p) x->p->r = x;
	if(is_left){
		if(b) b->p = p;
		p->l = b;
		p->p = x;
		x->r = p;
	}
	else{
		if(b) b->p = p;
		p->r = b;
		p->p = x;
		x->l = p;
	}
	pull(p); // if there's something to pull up
	pull(x);
	if(!x->p) root = x; // IF YOU ARE SPLAY TREE
	if(p->pp){ // IF YOU ARE LINK CUT TREE
		x->pp = p->pp;
		p->pp = NULL;
	}
}
void splay(node *x){
	while(x->p){
		node *p = x->p;
		node *g = p->p;
		if(g){
			if((p->l == x) ^ (g->l == p)) rotate(x);
			else rotate(p);
		}
		rotate(x);
	}
}
void access(node *x){
	splay(x);
	push(x);
	if(x->r){
		x->r->pp = x;
		x->r->p = NULL;
		x->r = NULL;
	}
	pull(x);
	while(x->pp){
		node *nxt = x->pp;
		splay(nxt);
		push(nxt);
		if(nxt->r){
			nxt->r->pp = nxt;
			nxt->r->p = NULL;
			nxt->r = NULL;
		}
		nxt->r = x;
		x->p = nxt;
		x->pp = NULL;
		pull(nxt);
		splay(x);
	}
}
node *root(node *x){
	access(x);
	while(x->l){
		push(x);
		x = x->l;
	}
	access(x);
	return x;
}
node *par(node *x){
	access(x);
	if(!x->l) return NULL;
	push(x);
	x = x->l;
	while(x->r){
		push(x);
		x = x->r;
	}
	access(x);
	return x;
}
node *lca(node *s, node *t){
	access(s);
	access(t);
	splay(s);
	if(s->pp == NULL) return s;
	return s->pp;
}
void link(node *par, node *son){
	access(par);
	access(son);
	son->rev ^= 1;  // remove if needed
	push(son);
	son->l = par;
	par->p = son;
	pull(son);
}
void cut(node *p){
	access(p);
	push(p);
	if(p->l){
		p->l->p = NULL;
		p->l = NULL;
	}
	pull(p);
}
\end{minted}
\subsection{Offline Dynamic MST}
\begin{minted}{cpp}
int n, m, q;
int st[MAXN], ed[MAXN], cost[MAXN], chk[MAXN];
pi qr[MAXN];

bool cmp(int &a, int &b){ return pi(cost[a], a) < pi(cost[b], b); }

void contract(int s, int e, vector<int> v, vector<int> &must_mst, vector<int> &maybe_mst){
	sort(v.begin(), v.end(), cmp);
	vector<pi> snapshot;
	for(int i=s; i<=e; i++) disj.uni(st[qr[i].first], ed[qr[i].first], snapshot);
	for(auto &i : v) if(disj.uni(st[i], ed[i], snapshot)) must_mst.push_back(i);
	disj.revert(snapshot);
	for(auto &i : must_mst) disj.uni(st[i], ed[i], snapshot);
	for(auto &i : v) if(disj.uni(st[i], ed[i], snapshot)) maybe_mst.push_back(i);
	disj.revert(snapshot);
}

void solve(int s, int e, vector<int> v, lint cv){
	if(s == e){
		cost[qr[s].first] = qr[s].second;
		if(st[qr[s].first] == ed[qr[s].first]){
			printf("%lld\n", cv);
			return;
		}
		int minv = qr[s].second;
		for(auto &i : v) minv = min(minv, cost[i]);
		printf("%lld\n",minv + cv);
		return;
	}
	int m = (s+e)/2;
	vector<int> lv = v, rv = v;
	vector<int> must_mst, maybe_mst;
	for(int i=m+1; i<=e; i++){
		chk[qr[i].first]--;
		if(chk[qr[i].first] == 0) lv.push_back(qr[i].first);
	}
	vector<pi> snapshot;
	contract(s, m, lv, must_mst, maybe_mst);
	lint lcv = cv;
	for(auto &i : must_mst) lcv += cost[i], disj.uni(st[i], ed[i], snapshot);
	solve(s, m, maybe_mst, lcv);
	disj.revert(snapshot);
	must_mst.clear(); maybe_mst.clear();
	for(int i=m+1; i<=e; i++) chk[qr[i].first]++;
	for(int i=s; i<=m; i++){
		chk[qr[i].first]--;
		if(chk[qr[i].first] == 0) rv.push_back(qr[i].first);
	}
	lint rcv = cv;
	contract(m+1, e, rv, must_mst, maybe_mst);
	for(auto &i : must_mst) rcv += cost[i], disj.uni(st[i], ed[i], snapshot);
	solve(m+1, e, maybe_mst, rcv);
	disj.revert(snapshot);
	for(int i=s; i<=m; i++) chk[qr[i].first]++;
}

int main(){
	scanf("%d %d",&n,&m);
	vector<int> ve;
	for(int i=0; i<m; i++){
		scanf("%d %d %d",&st[i],&ed[i],&cost[i]);
	}
	scanf("%d",&q);
	for(int i=0; i<q; i++){
		scanf("%d %d",&qr[i].first,&qr[i].second);
		qr[i].first--;
		chk[qr[i].first]++;
	}
	disj.init(n);
	for(int i=0; i<m; i++) if(!chk[i]) ve.push_back(i);
	solve(0, q-1, ve, 0);
}
\end{minted}
\subsection {Dominator Tree}
\begin{minted}{cpp}
vector<int> E[MAXN], RE[MAXN], rdom[MAXN];

int S[MAXN], RS[MAXN], cs;
int par[MAXN], val[MAXN], sdom[MAXN], rp[MAXN], dom[MAXN];

void clear(int n) {
	cs = 0;
	for(int i=0;i<=n;i++) {
		par[i] = val[i] = sdom[i] = rp[i] = dom[i] = S[i] = RS[i] = 0;
		E[i].clear(); RE[i].clear(); rdom[i].clear();
	}
}
void add_edge(int x, int y) { E[x].push_back(y); }
void Union(int x, int y) { par[x] = y; }
int Find(int x, int c = 0) {
	if(par[x] == x) return c ? -1 : x;
	int p = Find(par[x], 1);
	if(p == -1) return c ? par[x] : val[x];
	if(sdom[val[x]] > sdom[val[par[x]]]) val[x] = val[par[x]];
	par[x] = p;
	return c ? p : val[x];
}
void dfs(int x) {
	RS[ S[x] = ++cs ] = x;
	par[cs] = sdom[cs] = val[cs] = cs;
	for(int e : E[x]) {
		if(S[e] == 0) dfs(e), rp[S[e]] = S[x];
		RE[S[e]].push_back(S[x]);
	}
}
int solve(int s, int *up) { // Calculate idoms
	dfs(s);
	for(int i=cs;i;i--) {
		for(int e : RE[i]) sdom[i] = min(sdom[i], sdom[Find(e)]);
		if(i > 1) rdom[sdom[i]].push_back(i);
		for(int e : rdom[i]) {
			int p = Find(e);
			if(sdom[p] == i) dom[e] = i;
			else dom[e] = p;
		}
		if(i > 1) Union(i, rp[i]);
	}
	for(int i=2;i<=cs;i++) if(sdom[i] != dom[i]) dom[i] = dom[dom[i]];
	for(int i=2;i<=cs;i++) up[RS[i]] = RS[dom[i]];
	return cs;
}
\end{minted}
\subsection {Global Min-Cut}
\begin{minted}{cpp}
int minimum_cut_phase(int n, int &s, int &t, vector<vector<int>> &adj, vector<int> vis){
	vector<int> dist(n);
	int mincut = 1e9;
	while(true){
		int pos = -1, cur = -1e9;
		for(int i=0; i<n; i++){
			if(!vis[i] && dist[i] > cur){
				cur = dist[i];
				pos = i;
			}
		}
		if(pos == -1) break;
		s = t;
		t = pos;
		mincut = cur;
		vis[pos] = 1;
		for(int i=0; i<n; i++){
			if(!vis[i]) dist[i] += adj[pos][i];
		}
	}
	return mincut; // optimal s-t cut here is, {t} and V \ {t}
}
int solve(int n, vector<vector<int>> adj){
	if(n <= 1) return 0;
	vector<int> vis(n);
	int ans = 1e9;
	for(int i=0; i<n-1; i++){
		int s, t;
		ans = min(ans, minimum_cut_phase(n, s, t, adj, vis));
		vis[t] = 1;
		for(int j=0; j<n; j++){
			if(!vis[j]){
				adj[s][j] += adj[t][j];
				adj[j][s] += adj[j][t];
			}
		}
		adj[s][s] = 0;
	}
	return ans;
}
\end{minted}
\subsection {Edmond's Directed MST}
\revised
\begin{minted}{cpp}
// starts from node 0. assumes there exists at least one dmst.
// edge is reversed : if there is edge s -> e, INSERT IN gph[e]
struct edge { int to, cost, id; };
using elist = vector<edge>;
void dmst(vector<elist> &g, vector<int>& res) {
	const int n = g.size();
	vector<edge*> to(n);
	vector<int> u(n, 0);
	for (int i = 1; i < n; ++i) {
		int mn = g[i][0].cost;
		for (int j = 0; j < g[i].size(); ++j)
			mn = min(mn, g[i][j].cost);
		for (int j = 0; j < g[i].size(); ++j) {
			if (g[i][j].cost == mn)
				to[i] = &g[i][j];
			g[i][j].cost -= mn;
		}
	}
	for (int i = 1; i < n; ++i) {
		if (u[i]) continue;
		int x = i;
		vector<int> order(1, x);
		u[x] = 1;
		while (to[x]->to > 0 && u[to[x]->to] == 0) {
			x = to[x]->to;
			u[x] = 1;
			order.push_back(x);
		}
		int y = to[x]->to;
		vector<int> cycle(find(order.begin(), order.end(), y), order.end());
		if (cycle.size() == 0) continue;
		vector<int> in_cycle(n);
		for (int j = 0; j < cycle.size(); ++j) in_cycle[cycle[j]] = 1;
		vector<int> nw_id(n);
		for (int j = 1; j < n; ++j) nw_id[j] = nw_id[j - 1] + !in_cycle[j];
		int nn = n - cycle.size();
		vector<elist> gn(nn + 1);
		for (int j = 1; j < n; ++j) {
			if (in_cycle[j]) {
				for (int k = 0; k < g[j].size(); ++k)
					if (!in_cycle[g[j][k].to])
						gn[nn].push_back({nw_id[g[j][k].to], g[j][k].cost, g[j][k].id});
			} else {
				for (int k = 0; k < g[j].size(); ++k)
					if (in_cycle[g[j][k].to])
						gn[nw_id[j]].push_back({nn, g[j][k].cost, g[j][k].id});
					else
						gn[nw_id[j]].push_back({nw_id[g[j][k].to], g[j][k].cost, g[j][k].id});
			}
		}
		dmst(gn, res);
		set<int> used_e(res.begin(), res.end());
		for (int j = 0; j < cycle.size(); ++j) {
			bool found = false;
			for (int k = 0; k < g[cycle[j]].size(); ++k)
				found |= used_e.count(g[cycle[j]][k].id);
			if (found) {
				for (int k = 0; k < cycle.size(); ++k) {
					if (k != j)
						res.push_back(to[cycle[k]]->id);
				}
				return;
			}
		}
	}
	for (int i = 1; i < n; ++i) res.push_back(to[i]->id);
}
\end{minted}
\subsection {Vizing's Theorem}
\begin{minted}{cpp}
namespace Vizing{ // returns edge coloring in adjacent matrix G. 1 - based
	int C[MAXN][MAXN], G[MAXN][MAXN];
	void clear(int N){
		for(int i=0; i<=N; i++){
			for(int j=0; j<=N; j++) C[i][j] = G[i][j] = 0;
		}
	}
	void solve(vector<pi> &E, int N, int M){
		int X[MAXN] = {}, a;
		auto update = [&](int u){ for(X[u] = 1; C[u][X[u]]; X[u]++); };
		auto color = [&](int u, int v, int c){
			int p = G[u][v];
			G[u][v] = G[v][u] = c;
			C[u][c] = v; C[v][c] = u;
			C[u][p] = C[v][p] = 0;
			if( p ) X[u] = X[v] = p;
			else update(u), update(v);
			return p; 
		};
		auto flip = [&](int u, int c1, int c2){
			int p = C[u][c1];
			swap(C[u][c1], C[u][c2]);
			if( p ) G[u][p] = G[p][u] = c2;
			if( !C[u][c1] ) X[u] = c1;
			if( !C[u][c2] ) X[u] = c2;
			return p; 
		};
		for(int i = 1; i <= N; i++) X[i] = 1;
		for(int t = 0; t < E.size(); t++){
			int u = E[t].first, v0 = E[t].second, v = v0, c0 = X[u], c = c0, d;
			vector<pi> L;
			int vst[MAXN] = {};
			while(!G[u][v0]){
				L.emplace_back(v, d = X[v]);
				if(!C[v][c]) for(a = (int)L.size()-1; a >= 0; a--) c = color(u, L[a].first, c);
				else if(!C[u][d])for(a=(int)L.size()-1;a>=0;a--)color(u,L[a].first,L[a].second);
				else if( vst[d] ) break;
				else vst[d] = 1, v = C[u][d];
			}
			if( !G[u][v0] ){
				for(;v; v = flip(v, c, d), swap(c, d));
				if(C[u][c0]){
					for(a = (int)L.size()-2; a >= 0 && L[a].second != c; a--);
					for(; a >= 0; a--) color(u, L[a].first, L[a].second);
				} else t--;
			}
		}
	}
}
\end{minted}

\section{Strings}
\subsection{Aho-Corasick Algorithm}
\begin{minted}{cpp}
const int MAXN = 100005, MAXC = 26;
int trie[MAXN][MAXC], fail[MAXN], term[MAXN], piv;
void init(vector<string> &v){
	memset(trie, 0, sizeof(trie));
	memset(fail, 0, sizeof(fail));
	memset(term, 0, sizeof(term));
	piv = 0;
	for(auto &i : v){
		int p = 0;
		for(auto &j : i){
			if(!trie[p][j]) trie[p][j] = ++piv;
			p = trie[p][j];
		}
		term[p] = 1;
	}
	queue<int> que;
	for(int i=0; i<MAXC; i++){
		if(trie[0][i]) que.push(trie[0][i]);
	}
	while(!que.empty()){
		int x = que.front();
		que.pop();
		for(int i=0; i<MAXC; i++){
			if(trie[x][i]){
				int p = fail[x];
				while(p && !trie[p][i]) p = fail[p];
				p = trie[p][i];
				fail[trie[x][i]] = p;
				if(term[p]) term[trie[x][i]] = 1;
				que.push(trie[x][i]);
			}
		}
	}
}
bool query(string &s){
	int p = 0;
	for(auto &i : s){
		while(p && !trie[p][i]) p = fail[p];
		p = trie[p][i];
		if(term[p]) return 1;
	}
	return 0;
}
\end{minted}
\subsection{Suffix Array}
\begin{minted}{cpp}
const int MAXN = 500005;
int ord[MAXN], nord[MAXN], cnt[MAXN], aux[MAXN];
void solve(int n, char *str, int *sfx, int *rev, int *lcp){
	int p = 1;
	memset(ord, 0, sizeof(ord));
	for(int i=0; i<n; i++){
		sfx[i] = i;
		ord[i] = str[i];
	}
	int pnt = 1;
	while(1){
		memset(cnt, 0, sizeof(cnt));
		for(int i=0; i<n; i++) cnt[ord[min(i+p, n)]]++;
		for(int i=1; i<=n || i<=255; i++) cnt[i] += cnt[i-1];
		for(int i=n-1; i>=0; i--)
			aux[--cnt[ord[min(i+p, n)]]] = i;
		memset(cnt, 0, sizeof(cnt));
		for(int i=0; i<n; i++) cnt[ord[i]]++;
		for(int i=1; i<=n || i<=255; i++) cnt[i] += cnt[i-1];
		for(int i=n-1; i>=0; i--)
			sfx[--cnt[ord[aux[i]]]] = aux[i];
		if(pnt == n) break;
		pnt = 1;
		nord[sfx[0]] = 1;
		for(int i=1; i<n; i++){
			if(ord[sfx[i-1]] != ord[sfx[i]] || ord[sfx[i-1] + p] != ord[sfx[i] + p]){
				pnt++;
			}
			nord[sfx[i]] = pnt;
		}
		memcpy(ord, nord, sizeof(int) * n);
		p *= 2;
	}
	for(int i=0; i<n; i++) rev[sfx[i]] = i;
	int h = 0;
	for(int i=0; i<n; i++){
		if(rev[i]){
			int prv = sfx[rev[i] - 1];
			while(str[prv + h] == str[i + h]) h++;
			lcp[rev[i]] = h;
		}
		h = max(h-1, 0);
	}
}
\end{minted}
\subsection{Manacher's Algorithm}
\begin{minted}{cpp}
const int MAXN = 1000005;
int aux[2 * MAXN - 1];
void solve(int n, int *str, int *ret){
	// *ret : number of nonobvious palindromic character pair
	for(int i=0; i<n; i++){
		aux[2*i] = str[i];
		if(i != n-1) aux[2*i+1] = -1;
	}
	int p = 0, c = 0;
	for(int i=0; i<2*n-1; i++){
		int cur = 0;
		if(i <= p) cur = min(ret[2 * c - i], p - i);
		while(i - cur - 1 >= 0 && i + cur + 1 < 2*n-1 && aux[i-cur-1] == aux[i+cur+1]){
			cur++;
		}
		ret[i] = cur;
		if(i + ret[i] > p){
			p = i + ret[i];
			c = i;
		}
	}
}
\end{minted}

\subsection{Suffix Array (Linear time)}
\revised
\begin{minted}{cpp}
class SuffixArray {
public:
	int A[7 * N / 10], B[7 * N / 10], cnt[N + 2], SAV[N];
	int mem[5 * N]; int* mem_pt = mem;
	void clear(int n){
		int *ptr = mem;
		while(ptr != mem_pt){
			*ptr = 0;
			ptr++;
		}
		mem_pt = mem;
		for(int i=0; i<n+10 && i < 7 * N / 10; i++) A[i] = B[i] = 0;
		for(int i=0; i<n+2; i++) cnt[i] = 0;
		for(int i=0; i<n; i++) SAV[i] = 0;
	}
	inline int* mloc(size_t sz) {
		int* ret = mem_pt; mem_pt = mem_pt + sz;
		return ret;
	}
	void rsort(int* a, int* b, int* dat, int n, int k) { 
		for (int i = 0; i <= k; i++) cnt[i] = 0;
		for (int i = 0; i < n; i++) SAV[i] = dat[a[i]], cnt[SAV[i]]++;
		for (int i = 1; i <= k; i++) cnt[i] += cnt[i - 1];
		for (int i = n - 1; i >= 0; i--) b[--cnt[SAV[i]]] = a[i];
	}

#define I(x) ((x)%3==1)?((x)/3):((x)/3+num1)
#define I2(x) (x<num1)?(3*x+1):(3*(x-num1)+2)
	static int cmp(int x, int y, int str[], int A[], int num1) {
		if (x % 3 == 1) {
			if (y % 3 == 2) return A[I(x)] < A[I(y)];
			else return str[x] < str[y] || str[x] == str[y] && A[I(x + 1)] < A[I(y + 1)];
		}
		else {
			return str[x] < str[y] || str[x] == str[y] && cmp(x + 1, y + 1, str, A, num1);
		}
	}
	void make(int* str, int* sa, int n, int k) {
		if (n == 0) return;
		int num1 = (n + 2) / 3, num2 = n / 3;
		int num = num1 + num2;
		str[n] = str[n + 1] = str[n + 2] = 0;
		int *nsa = mloc(num), *nstr = mloc(num + 3);

		for (int i = 0, j = 0; i < n; i++) if (i % 3) A[j++] = i;
		if (n % 3 == 1) A[num - 1] = n;
		rsort(A, B, str + 2, num, k); rsort(B, A, str + 1, num, k); rsort(A, B, str, num, k);

		int cnt = 1;
		nstr[I(B[0])] = 1;
		for (int i = 1; i < num; i++) {
			int c = B[i], p = B[i - 1];
			if (str[p] != str[c] || str[p + 1] != str[c + 1] || str[p + 2] != str[c + 2]) cnt++;
			nstr[I(c)] = cnt;
		}
		if (cnt == num) for (int i = 0; i < num; i++) nsa[nstr[i] - 1] = i;
		else make(nstr, nsa, num, cnt);
		
		for (int i = 0, j = 0; i < num; i++) if (nsa[i] < num1) A[j++] = 3 * nsa[i];
		rsort(A, B, str, num1, k);
		for (int i = 0; i < num; i++) A[nsa[i]] = i, nsa[i] = I2(nsa[i]);
		A[num] = -1;
		merge(B, B + num1, nsa + (n % 3 == 1), nsa + num, sa, [&](int x, int y) {
			return cmp(x, y, str, A, num1); 
		});
		return;
	}
}sa;
\end{minted}

\subsection{eertree}
\begin{minted}{cpp}
int nxt[MAXN][26];
int par[MAXN], len[MAXN], slink[MAXN], ptr[MAXN], diff[MAXN], series[MAXN], piv;
void clear(int n = MAXN){
	memset(par, 0, sizeof(int) * n);
	memset(len, 0, sizeof(int) * n);
	memset(slink, 0, sizeof(int) * n);
	memset(nxt, 0, sizeof(int) * 26 * n);
	piv = 0;
}
void init(int n, char *a){
	par[0] = 0;
	par[1] = 1;
	a[0] = -1;
	len[0] = -1;
	piv = 1;
	int cur = 1;
	for(int i=1; i<=n; i++){
		while(a[i] != a[i - len[cur] - 1]) cur = slink[cur];
		if(!nxt[cur][a[i]]){
			nxt[cur][a[i]] = ++piv;
			par[piv] = cur;
			len[piv] = len[cur] + 2;
			int lnk = slink[cur];
			while(a[i] != a[i - len[lnk] - 1]){
				lnk = slink[lnk];
			}
			if(nxt[lnk][a[i]]) lnk = nxt[lnk][a[i]];
			if(len[piv] == 1 || lnk == 0) lnk = 1;
			slink[piv] = lnk;
			diff[piv] = len[piv] - len[lnk];
			if(diff[piv] == diff[lnk]) series[piv] = series[lnk];
			else series[piv] = piv;
		}
		cur = nxt[cur][a[i]];
		ptr[i] = cur;
	}
}
int query(int s, int e){
	int pos = ptr[e];
	while(len[pos] >= e - s + 1){
        if(len[pos] % diff[pos] == (e - s + 1) % diff[pos] &&
           len[series[pos]] <= e - s + 1) return true;
        pos = series[pos];
        pos = slink[pos];
	}
	return false;
}
vector<pi> minimum_partition(int n){ // (odd min, even min)
	vector<pi> dp(n + 1);
	vector<pi> series_ans(n + 10);
	dp[0] = pi(1e9 + 1, 0);
	for(int i=1; i<=n; i++){
		dp[i] = pi(1e9 + 1, 1e9);
		for(int j=ptr[i]; len[j] > 0;){
			int slv = slink[series[j]];
			series_ans[j] = dp[i - (len[slv] + diff[j])];
			if(diff[j] == diff[slink[j]]){
				series_ans[j].first = min(series_ans[j].first, series_ans[slink[j]].first);
				series_ans[j].second = min(series_ans[j].second, series_ans[slink[j]].second);
			}
			auto val = series_ans[j];
			dp[i].first = min(dp[i].first, val.second + 1);
			dp[i].second = min(dp[i].second, val.first + 1);
			j = slv;
		}
	}
	return dp;
}
\end{minted}
\subsection{Circular LCS}
\begin{minted}{cpp}
string s1, s2;
int dp[4005][2005];
int nxt[4005][2005];
int n, m;
void reroot(int px){
	int py = 1;
	while(py <= m && nxt[px][py] != 2) py++;
	nxt[px][py] = 1;
	while(px < 2 * n && py < m){
		if(nxt[px+1][py] == 3){
			px++;
			nxt[px][py] = 1;
		}
		else if(nxt[px+1][py+1] == 2){
			px++;
			py++;
			nxt[px][py] = 1;
		}
		else py++;
	}
	while(px < 2 * n && nxt[px+1][py] == 3){
		px++;
		nxt[px][py] = 1;
	}
}

int track(int x, int y, int e){ // use this routine to find LCS as string
	int ret = 0;
	while(y != 0 && x != e){
		if(nxt[x][y] == 1) y--;
		else if(nxt[x][y] == 2) ret += (s1[x] == s2[y]), x--, y--;
		else if(nxt[x][y] == 3) x--;
	}
	return ret;
}

int solve(string a, string b){
	n = a.size(), m = b.size();
	s1 = "#" + a + a;
	s1 = '#' + b;
	for(int i=0; i<=2*n; i++){
		for(int j=0; j<=m; j++){
			if(j == 0){
				nxt[i][j] = 3;
				continue;
			}
			if(i == 0){
				nxt[i][j] = 1;
				continue;
			}
			dp[i][j] = -1;
			if(dp[i][j] < dp[i][j-1]){
				dp[i][j] = dp[i][j-1];
				nxt[i][j] = 1;
			}
			if(dp[i][j] < dp[i-1][j-1] + (s1[i] == s2[j])){
				dp[i][j] = dp[i-1][j-1] + (s1[i] == s2[j]);
				nxt[i][j] = 2;
			}
			if(dp[i][j] < dp[i-1][j]){
				dp[i][j] = dp[i-1][j];
				nxt[i][j] = 3;
			}
		}
	}
	int ret = dp[n][m];
	for(int i=1; i<n; i++){
		reroot(i), ret = max(ret, track(n+i, m, i));
	}
	return ret;
}
\end{minted}
\section{Geometry}
\subsection{Smallest Enclosing Circle / Sphere}
\begin{minted}{cpp}
namespace cover_2d{
	double eps = 1e-9;
	using Point = complex<double>;
	struct Circle{ Point p; double r; };
	double dist(Point p, Point q){ return abs(p-q); }
	double area2(Point p, Point q){ return (conj(p)*q).imag(); }
	bool in(const Circle& c, Point p){ return dist(c.p, p) < c.r + eps; }
	Circle INVAL = Circle{Point(0, 0), -1};
	Circle mCC(Point a, Point b, Point c){
		b -= a; c -= a;
		double d = 2*(conj(b)*c).imag(); if(abs(d)<eps) return INVAL;
		Point ans = (c*norm(b) - b*norm(c)) * Point(0, -1) / d;
		return Circle{a + ans, abs(ans)};
	}
	Circle solve(vector<Point> p) {
		mt19937 gen(0x94949); shuffle(p.begin(), p.end(), gen);
		Circle c = INVAL;
		for(int i=0; i<p.size(); ++i) if(c.r<0 ||!in(c, p[i])){
			c = Circle{p[i], 0};
			for(int j=0; j<=i; ++j) if(!in(c, p[j])){
				Circle ans{(p[i]+p[j])*0.5, dist(p[i], p[j])*0.5};
				if(c.r == 0) { c = ans; continue; }
				Circle l, r; l = r = INVAL;
				Point pq = p[j]-p[i];
				for(int k=0; k<=j; ++k) if(!in(ans, p[k])) {
					double a2 = area2(pq, p[k]-p[i]);
					Circle c = mCC(p[i], p[j], p[k]);
					if(c.r<0) continue;
					else if(a2 > 0 && (l.r<0||area2(pq, c.p-p[i]) > area2(pq, l.p-p[i]))) l = c;
					else if(a2 < 0 && (r.r<0||area2(pq, c.p-p[i]) < area2(pq, r.p-p[i]))) r = c;
				}
				if(l.r<0&&r.r<0) c = ans;
				else if(l.r<0) c = r;
				else if(r.r<0) c = l;
				else c = l.r<=r.r?l:r;
			}
		}
		return c;
	}
};

namespace cover_3d{
	double enclosing_sphere(vector<double> x, vector<double> y, vector<double> z){
		int n = x.size();
		auto hyp = [](double x, double y, double z){
			return x * x + y * y + z * z;
		};
		double px = 0, py = 0, pz = 0;
		for(int i=0; i<n; i++){
			px += x[i];
			py += y[i];
			pz += z[i];
		}
		px *= 1.0 / n;
		py *= 1.0 / n;
		pz *= 1.0 / n;
		double rat = 0.1, maxv;
		for(int i=0; i<10000; i++){
			maxv = -1;
			int maxp = -1;
			for(int j=0; j<n; j++){
				double tmp = hyp(x[j] - px, y[j] - py, z[j] - pz);
				if(maxv < tmp){
					maxv = tmp;
					maxp = j;
				}
			}
			px += (x[maxp] - px) * rat;
			py += (y[maxp] - py) * rat;
			pz += (z[maxp] - pz) * rat;
			rat *= 0.998;
		}
		return sqrt(maxv);
	}
};
\end{minted}
\subsection{3D Convex Hull}
\begin{minted}{cpp}
struct vec3{
	ll x, y, z;
	vec3(): x(0), y(0), z(0) {}
	vec3(ll a, ll b, ll c): x(a), y(b), z(c) {}
	vec3 operator*(const vec3& v) const{ return vec3(y*v.z-z*v.y, z*v.x-x*v.z, x*v.y-y*v.x); }
	vec3 operator-(const vec3& v) const{ return vec3(x-v.x, y-v.y, z-v.z); }
	vec3 operator-() const{ return vec3(-x, -y, -z); }
	ll dot(const vec3 &v) const{ return x*v.x+y*v.y+z*v.z; }
};

struct twoset {
	int a, b;
	void insert(int x) { (a == -1 ? a : b) = x; }
	bool contains(int x) { return a == x || b == x; }
	void erase(int x) { (a == x ? a : b) = -1; }
	int size() { return (a != -1) + (b != -1); }
} E[MAXN][MAXN]; // i < j

struct face{
	vec3 norm;
	ll disc;
	int I[3];
};

face make_face(int i, int j, int k, int ii, vector<vec3> &A){ // p^T * norm < disc
	E[i][j].insert(k); E[i][k].insert(j); E[j][k].insert(i);
	face f; f.I[0]=i, f.I[1]=j, f.I[2]=k;
	f.norm = (A[j]-A[i])*(A[k]-A[i]);
	f.disc = f.norm.dot(A[i]);
	if(f.norm.dot(A[ii])>f.disc){
		f.norm = -f.norm;
		f.disc = -f.disc;
	}
	return f;
}

vector<face> get_hull(vector<vec3> &A){
	int N = A.size();
	vector<face> faces; memset(E, -1, sizeof(E));
	faces.push_back(make_face(0,1,2,3,A));
	faces.push_back(make_face(0,1,3,2,A));
	faces.push_back(make_face(0,2,3,1,A));
	faces.push_back(make_face(1,2,3,0,A));
	for(int i=4; i<N; ++i){
		for(int j=0; j<faces.size(); ++j){
			face f = faces[j];
			if(f.norm.dot(A[i])>f.disc){
				E[f.I[0]][f.I[1]].erase(f.I[2]);
				E[f.I[0]][f.I[2]].erase(f.I[1]);
				E[f.I[1]][f.I[2]].erase(f.I[0]);
				faces[j--] = faces.back();
				faces.pop_back();
			}
		}
		int nf = faces.size();
		for(int j=0; j<nf; ++j){
			face f=faces[j];
			for(int a=0; a<3; ++a) for(int b=a+1; b<3; ++b){
				int c=3-a-b;
				if(E[f.I[a]][f.I[b]].size()==2) continue;
				faces.push_back(make_face(f.I[a], f.I[b], i, f.I[c], A));
			}
		}
	}
	return faces;
}
\end{minted}
\subsection{Dynamic Convex Hull Trick}
\begin{minted}{cpp}
using line_t = double;
const line_t is_query = -1e18;

struct Line {
	line_t m, b;
	mutable function<const Line*()> succ;
	bool operator<(const Line& rhs) const {
		if (rhs.b != is_query) return m < rhs.m;
		const Line* s = succ();
		if (!s) return 0;
		line_t x = rhs.m;
		return b - s->b < (s->m - m) * x;
	}
};

struct HullDynamic : public multiset<Line> { // will maintain upper hull for maximum
	bool bad(iterator y) {
		auto z = next(y);
		if (y == begin()) {
			if (z == end()) return 0;
			return y->m == z->m && y->b <= z->b;
		}
		auto x = prev(y);
		if (z == end()) return y->m == x->m && y->b <= x->b;
		return (x->b - y->b)*(z->m - y->m) >= (y->b - z->b)*(y->m - x->m);
	}
	void insert_line(line_t m, line_t b) {
		auto y = insert({ m, b });
		y->succ = [=] { return next(y) == end() ? 0 : &*next(y); };
		if (bad(y)) { erase(y); return; }
		while (next(y) != end() && bad(next(y))) erase(next(y));
		while (y != begin() && bad(prev(y))) erase(prev(y));
	}
	line_t query(line_t x) {
		auto l = *lower_bound((Line) { x, is_query });
		return l.m * x + l.b;
	}
}H;
\end{minted}
\subsection{Half-plane Intersection}
\begin{minted}{cpp}
const double eps = 1e-8;
typedef pair<long double, long double> pi;
bool z(long double x){ return fabs(x) < eps; }
struct line{
	long double a, b, c;
	bool operator<(const line &l)const{
		bool flag1 = pi(a, b) > pi(0, 0);
		bool flag2 = pi(l.a, l.b) > pi(0, 0);
		if(flag1 != flag2) return flag1 > flag2;
		long double t = ccw(pi(0, 0), pi(a, b), pi(l.a, l.b));
		return z(t) ? c * hypot(l.a, l.b) < l.c * hypot(a, b) : t > 0;
	}
	pi slope(){ return pi(a, b); }
};
pi cross(line a, line b){
	long double det = a.a * b.b - b.a * a.b;
	return pi((a.c * b.b - a.b * b.c) / det, (a.a * b.c - a.c * b.a) / det);
}
bool bad(line a, line b, line c){
	if(ccw(pi(0, 0), a.slope(), b.slope()) <= 0) return false;
	pi crs = cross(a, b);
	return crs.first * c.a + crs.second * c.b >= c.c;
}
bool solve(vector<line> v, vector<pi> &solution){ // ax + by <= c;
	sort(v.begin(), v.end());
	deque<line> dq;
	for(auto &i : v){
		if(!dq.empty() && z(ccw(pi(0, 0), dq.back().slope(), i.slope()))) continue;
		while(dq.size() >= 2 && bad(dq[dq.size()-2], dq.back(), i)) dq.pop_back();
		while(dq.size() >= 2 && bad(i, dq[0], dq[1])) dq.pop_front();
		dq.push_back(i);
	}
	while(dq.size() > 2 && bad(dq[dq.size()-2], dq.back(), dq[0])) dq.pop_back();
	while(dq.size() > 2 && bad(dq.back(), dq[0], dq[1])) dq.pop_front();
	vector<pi> tmp;
	for(int i=0; i<dq.size(); i++){
		line cur = dq[i], nxt = dq[(i+1)%dq.size()];
		if(ccw(pi(0, 0), cur.slope(), nxt.slope()) <= eps) return false;
		tmp.push_back(cross(cur, nxt));
	}
	solution = tmp;
	return true;
}
\end{minted}
\subsection{Point-in-polygon test / Point-to-polygon tangent}
\begin{minted}{cpp}
// C : counter_clockwise(C[0] == C[N]), N >= 3
// return highest point in C <- P(clockwise) or -1 if strictly in P
// polygon is strongly convex, C[i] != P
int convex_tangent(vector<pi> &C, pi P, int up = 1){
	auto sign = [&](lint c){ return c > 0 ? up : c == 0 ? 0 : -up; };
	auto local = [&](pi P, pi a, pi b, pi c) {
		return sign(ccw(P, a, b)) <= 0 && sign(ccw(P, b, c)) >= 0;
	};
	int N = C.size()-1, s = 0, e = N, m;
	if( local(P, C[1], C[0], C[N-1]) ) return 0;
	while(s+1 < e){
		m = (s+e) / 2;
		if( local(P, C[m-1], C[m], C[m+1]) ) return m;
		if( sign(ccw(P, C[s], C[s+1])) < 0){ // up
			if( sign(ccw(P, C[m], C[m+1])) > 0) e = m;
			else if( sign(ccw(P, C[m], C[s])) > 0) s = m;
			else e = m;
		}
		else{ // down
			if( sign(ccw(P, C[m], C[m+1])) < 0) s = m;
			else if( sign(ccw(P, C[m], C[s])) < 0) s = m;
			else e = m;
		}
	}
	if( s && local(P, C[s-1], C[s], C[s+1]) ) return s;
	if( e != N && local(P, C[e-1], C[e], C[e+1]) ) return e;
	return -1;
}
\end{minted}
\subsection{kd-tree}
\begin{minted}{cpp}
typedef pair<int, int> pi;
struct node{
	pi pnt;
	int spl, sx, ex, sy, ey;
}tree[270000];

pi a[100005];
int n, ok[270000];

lint sqr(int x){ return 1ll * x * x; }
bool cmp1(pi a, pi b){ return a < b; }
bool cmp2(pi a, pi b){ return pi(a.second, a.first) < pi(b.second, b.first); }

// init(0, n-1, 1) : Initialize kd-tree
// set dap = INF, and call solve(1, P). dap = (closest point from P)
void init(int s, int e, int p){ // Initialize kd-tree
	int minx = 1e9, maxx = -1e9, miny = 1e9, maxy = -1e9;
	int m = (s+e)/2;
	for(int i=s; i<=e; i++){
		minx = min(minx, a[i].first);
		miny = min(miny, a[i].second);
		maxx = max(maxx, a[i].first);
		maxy = max(maxy, a[i].second);
	}
	tree[p].spl = (maxx - minx < maxy - miny);
	sort(a+s, a+e+1, [&](const pi &a, const pi &b){
		return tree[p].spl ? cmp2(a, b) : cmp1(a, b);
	});
	ok[p] = 1;
	tree[p] = {a[m], tree[p].spl, minx, maxx, miny, maxy};
	if(s <= m-1) init(s, m-1, 2*p);
	if(m+1 <= e) init(m+1, e, 2*p+1);
}

lint dap = 3e18;

void solve(int p, pi x){ // find closest point from point x (L^2)
	if(x != tree[p].pnt) dap = min(dap, sqr(x.first - tree[p].pnt.first) + sqr(x.second - tree[p].pnt.second));
	if(tree[p].spl){
		if(!cmp2(tree[p].pnt, x)){
			if(ok[2*p]) solve(2*p, x);
			if(ok[2*p+1] && sqr(tree[2*p+1].sy - x.second) < dap) solve(2*p+1, x);
		}
		else{
			if(ok[2*p+1]) solve(2*p+1, x);
			if(ok[2*p] && sqr(tree[2*p].ey - x.second) < dap) solve(2*p, x);
		}
	}
	else{
		if(!cmp1(tree[p].pnt, x)){
			if(ok[2*p]) solve(2*p, x);
			if(ok[2*p+1] && sqr(tree[2*p+1].sx - x.first) < dap) solve(2*p+1, x);
		}
		else{
			if(ok[2*p+1]) solve(2*p+1, x);
			if(ok[2*p] && sqr(tree[2*p].ex - x.first) < dap) solve(2*p, x);
		}
	}
}
\end{minted}
\section{Math}
\subsection{FFT / NTT}
\begin{minted}{cpp}
typedef complex<double> base;
void fft(vector<base> &a, bool inv){
	int n = a.size(), j = 0;
	vector<base> roots(n/2);
	for(int i=1; i<n; i++){
		int bit = (n >> 1);
		while(j >= bit){
			j -= bit;
			bit >>= 1;
		}
		j += bit;
		if(i < j) swap(a[i], a[j]);
	}
	double ang = 2 * acos(-1) / n * (inv ? -1 : 1);
	for(int i=0; i<n/2; i++){
		roots[i] = base(cos(ang * i), sin(ang * i));
	}
	/* In NTT, let prr = primitive root. Then,
	int ang = ipow(prr, (mod - 1) / n);
	if(inv) ang = ipow(ang, mod - 2);
	for(int i=0; i<n/2; i++){
		roots[i] = (i ? (1ll * roots[i-1] * ang % mod) : 1);
	}
	XOR Convolution : set roots[*] = 1.
	OR Convolution : set roots[*] = 1, and do following:
    if (!inv) {
        a[j + k] = u + v;
        a[j + k + i/2] = u;
    } else {
        a[j + k] = v;
        a[j + k + i/2] = u - v;
    }
	*/
	for(int i=2; i<=n; i<<=1){
		int step = n / i;
		for(int j=0; j<n; j+=i){
			for(int k=0; k<i/2; k++){
				base u = a[j+k], v = a[j+k+i/2] * roots[step * k];
				a[j+k] = u+v;
				a[j+k+i/2] = u-v;
			}
		}
	}
	if(inv) for(int i=0; i<n; i++) a[i] /= n; // skip for OR convolution.
}

vector<lint> multiply(vector<lint> &v, vector<lint> &w){
	vector<base> fv(v.begin(), v.end()), fw(w.begin(), w.end());
	int n = 2; while(n < v.size() + w.size()) n <<= 1;
	fv.resize(n); fw.resize(n);
	fft(fv, 0); fft(fw, 0);
	for(int i=0; i<n; i++) fv[i] *= fw[i];
	fft(fv, 1);
	vector<lint> ret(n);
	for(int i=0; i<n; i++) ret[i] = (lint)round(fv[i].real());
	return ret;
}
vector<lint> multiply(vector<lint> &v, vector<lint> &w, lint mod){
	int n = 2; while(n < v.size() + w.size()) n <<= 1;
	vector<base> v1(n), v2(n), r1(n), r2(n);
	for(int i=0; i<v.size(); i++){
		v1[i] = base(v[i] >> 15, v[i] & 32767);
	}
	for(int i=0; i<w.size(); i++){
		v2[i] = base(w[i] >> 15, w[i] & 32767);
	}
	fft(v1, 0);
	fft(v2, 0);
	for(int i=0; i<n; i++){
		int j = (i ? (n - i) : i);
		base ans1 = (v1[i] + conj(v1[j])) * base(0.5, 0);
		base ans2 = (v1[i] - conj(v1[j])) * base(0, -0.5);
		base ans3 = (v2[i] + conj(v2[j])) * base(0.5, 0);
		base ans4 = (v2[i] - conj(v2[j])) * base(0, -0.5);
		r1[i] = (ans1 * ans3) + (ans1 * ans4) * base(0, 1);
		r2[i] = (ans2 * ans3) + (ans2 * ans4) * base(0, 1);
	}
	fft(r1, 1);
	fft(r2, 1);
	vector<lint> ret(n);
	for(int i=0; i<n; i++){
		lint av = (lint)round(r1[i].real());
		lint bv = (lint)round(r1[i].imag()) + (lint)round(r2[i].real());
		lint cv = (lint)round(r2[i].imag());
		av %= mod, bv %= mod, cv %= mod;
		ret[i] = (av << 30) + (bv << 15) + cv;
		ret[i] %= mod;
		ret[i] += mod;
		ret[i] %= mod;
	}
	return ret;
}
\end{minted}
\subsection{Hell-Joseon style FFT}
\begin{minted}{cpp}
#include <smmintrin.h>
#include <immintrin.h>
#pragma GCC target("avx2")
#pragma GCC target("fma")
__m256d mult(__m256d a, __m256d b){
	__m256d c = _mm256_movedup_pd(a);
	__m256d d = _mm256_shuffle_pd(a, a, 15);
	__m256d cb = _mm256_mul_pd(c, b);
	__m256d db = _mm256_mul_pd(d, b);
	__m256d e = _mm256_shuffle_pd(db, db, 5);
	__m256d r = _mm256_addsub_pd(cb, e);
	return r;
}
void fft(int n, __m128d a[], bool invert){
	for(int i=1, j=0; i<n; ++i){
		int bit = n>>1;
		for(;j>=bit;bit>>=1) j -= bit;
		j += bit;
		if(i<j) swap(a[i], a[j]);
	}
	for(int len=2; len<=n; len<<=1){
		double ang = 2*3.14159265358979/len*(invert?-1:1);
		__m256d wlen; wlen[0] = cos(ang), wlen[1] = sin(ang);
		for(int i=0; i<n; i += len){
			__m256d w; w[0] = 1; w[1] = 0;
			for(int j=0; j<len/2; ++j){
				w = _mm256_permute2f128_pd(w, w, 0);
				wlen = _mm256_insertf128_pd(wlen, a[i+j+len/2], 1);
				w = mult(w, wlen);
				__m128d vw = _mm256_extractf128_pd(w, 1);
				__m128d u = a[i+j];
				a[i+j] = _mm_add_pd(u, vw);
				a[i+j+len/2] = _mm_sub_pd(u, vw);
			}
		}
	}
	if(invert){
		__m128d inv; inv[0] = inv[1] = 1.0/n;
		for(int i=0; i<n; ++i) a[i] = _mm_mul_pd(a[i], inv);
	}
}
vector<int64_t> multiply(vector<int64_t>& v, vector<int64_t>& w){
	int n = 2; while(n < v.size()+w.size()) n<<=1;
	__m128d* fv = new __m128d[n];
	for(int i=0; i<n; ++i) fv[i][0] = fv[i][1] = 0;
	for(int i=0; i<v.size(); ++i) fv[i][0] = v[i];
	for(int i=0; i<w.size(); ++i) fv[i][1] = w[i];
	fft(n, fv, 0); // (a+bi) is stored in FFT
	for(int i=0; i<n; i += 2){
		__m256d a;
		a = _mm256_insertf128_pd(a, fv[i], 0);
		a = _mm256_insertf128_pd(a, fv[i+1], 1);
		a = mult(a, a);
		fv[i] = _mm256_extractf128_pd(a, 0);
		fv[i+1] = _mm256_extractf128_pd(a, 1);
	}
	fft(n, fv, 1);
	vector<int64_t> ret(n);
	for(int i=0; i<n; ++i) ret[i] = (int64_t)round(fv[i][1]/2);
	delete[] fv;
	return ret;
}
\end{minted}
\subsection{NTT Polynomial Division}
\begin{minted}{cpp}
vector<lint> get_inv(int n, const vector<lint> &p){
	vector<lint> q = {ipow(p[0], mod - 2)};
	for(int i=2; i<=n; i<<=1){
		vector<lint> res;
		vector<lint> fq(q.begin(), q.end()); fq.resize(2*i);
		vector<lint> fp(p.begin(), p.begin() + i); fp.resize(2*i);
		fft(fq, 0); fft(fp, 0);
		for(int j=0; j<2*i; j++){
			fp[j] *= fq[j] * fq[j] % mod;
			fp[j] %= mod;
		}
		fft(fp, 1);
		res.resize(i);
		for(int j=0; j<i; j++){
			res[j] = mod - fp[j];
			if(j < i/2) res[j] += 2 * q[j];
			res[j] %= mod;
		}
		q = res;
	}
	return q;
}
vector<lint> poly_divide(const vector<lint> &a, const vector<lint> &b){
	assert(b.back() != 0); // please trim leading zero
	int n = a.size(), m = b.size();
	int k = 2; while(k < n-m+1) k <<= 1;
	vector<lint> rb(k), ra(k);
	for(int i=0; i<m && i<k; ++i) rb[i] = b[m-i-1];
	for(int i=0; i<n && i<k; ++i) ra[i] = a[n-i-1];
	vector<lint> rbi = get_inv(k, rb);
	vector<lint> res = multiply(rbi, ra);
	res.resize(n - m + 1);
	reverse(res.begin(), res.end());
	return res;
}
\end{minted}
\subsection{Black Box Linear Algebra + Kitamasa}
\begin{minted}{cpp}
vector<int> berlekamp_massey(vector<int> x){
	vector<int> ls, cur;
	int lf, ld;
	for(int i=0; i<x.size(); i++){
		lint t = 0;
		for(int j=0; j<cur.size(); j++){
			t = (t + 1ll * x[i-j-1] * cur[j]) % mod;
		}
		if((t - x[i]) % mod == 0) continue;
		if(cur.empty()){
			cur.resize(i+1);
			lf = i;
			ld = (t - x[i]) % mod;
			continue;
		}
		lint k = -(x[i] - t) * ipow(ld, mod - 2) % mod;
		vector<int> c(i-lf-1);
		c.push_back(k);
		for(auto &j : ls) c.push_back(-j * k % mod);
		if(c.size() < cur.size()) c.resize(cur.size());
		for(int j=0; j<cur.size(); j++){
			c[j] = (c[j] + cur[j]) % mod;
		}
		if(i-lf+(int)ls.size()>=(int)cur.size()){
			tie(ls, lf, ld) = make_tuple(cur, i, (t - x[i]) % mod);
		}
		cur = c;
	}
	for(auto &i : cur) i = (i % mod + mod) % mod;
	return cur;
}
int get_nth(vector<int> rec, vector<int> dp, lint n){
	int m = rec.size();
	vector<int> s(m), t(m);
	s[0] = 1;
	if(m != 1) t[1] = 1;
	else t[0] = rec[0];
	auto mul = [&rec](vector<int> v, vector<int> w){
		int m = v.size();
		vector<int> t(2 * m);
		for(int j=0; j<m; j++){
			for(int k=0; k<m; k++){
				t[j+k] += 1ll * v[j] * w[k] % mod;
				if(t[j+k] >= mod) t[j+k] -= mod;
			}
		}
		for(int j=2*m-1; j>=m; j--){
			for(int k=1; k<=m; k++){
				t[j-k] += 1ll * t[j] * rec[k-1] % mod;
				if(t[j-k] >= mod) t[j-k] -= mod;
			}
		}
		t.resize(m);
		return t;
	};
	while(n){
		if(n & 1) s = mul(s, t);
		t = mul(t, t);
		n >>= 1;
	}
	lint ret = 0;
	for(int i=0; i<m; i++) ret += 1ll * s[i] * dp[i] % mod;
	return ret % mod;
}
int guess_nth_term(vector<int> x, lint n){
	if(n < x.size()) return x[n];
	vector<int> v = berlekamp_massey(x);
	if(v.empty()) return 0;
	return get_nth(v, x, n);
}
struct elem{int x, y, v;}; // A_(x, y) <- v, 0-based. no duplicate please..
vector<int> get_min_poly(int n, vector<elem> M){ 
	// smallest poly P such that A^i = sum_{j < i} {A^j \times P_j} 
	vector<int> rnd1, rnd2;
	mt19937 rng(0x14004);
	auto randint = [&rng](int lb, int ub){
		return uniform_int_distribution<int>(lb, ub)(rng);
	};
	for(int i=0; i<n; i++){
		rnd1.push_back(randint(1, mod - 1));
		rnd2.push_back(randint(1, mod - 1));
	}
	vector<int> gobs;
	for(int i=0; i<2*n+2; i++){
		int tmp = 0;
		for(int j=0; j<n; j++){
			tmp += 1ll * rnd2[j] * rnd1[j] % mod;
			if(tmp >= mod) tmp -= mod;
		}
		gobs.push_back(tmp);
		vector<int> nxt(n);
		for(auto &i : M){
			nxt[i.x] += 1ll * i.v * rnd1[i.y] % mod;
			if(nxt[i.x] >= mod) nxt[i.x] -= mod;
		}
		rnd1 = nxt;
	}
	auto sol = berlekamp_massey(gobs);
	reverse(sol.begin(), sol.end());
	return sol;
}
lint det(int n, vector<elem> M){ 
	vector<int> rnd;
	mt19937 rng(0x14004);
	auto randint = [&rng](int lb, int ub){
		return uniform_int_distribution<int>(lb, ub)(rng);
	};
	for(int i=0; i<n; i++) rnd.push_back(randint(1, mod - 1));
	for(auto &i : M){
		i.v = 1ll * i.v * rnd[i.y] % mod;
	}
	auto sol = get_min_poly(n, M)[0];
	if(n % 2 == 0) sol = mod - sol;
	for(auto &i : rnd) sol = 1ll * sol * ipow(i, mod - 2) % mod;
	return sol;
}
\end{minted}
\subsection{Gaussian Elimination}
\begin{minted}{cpp}
int n, inv;
vector<int> basis[505];
lint gyesu = 1;

void insert(vector<int> v){
	for(int i=0; i<n; i++){
		if(basis[i].size()) inv ^= 1; // inversion num increases
		if(v[i] && basis[i].empty()){
			basis[i] = v;
			return;
		}
		if(v[i]){
			lint minv = ipow(basis[i][i], mod - 2) * v[i] % mod;
			for(auto &j : basis[i]) j = (j * minv) % mod;
			gyesu *= minv;
			gyesu %= mod;
			for(int j=0; j<basis[i].size(); j++){
				v[j] += mod - basis[i][j];
				while(v[j] >= mod) v[j] -= mod;
			}
		}
	}
	puts("0");
	exit(0);
}

// Sample: Calculates Determinant in Z_p Field
int main(){
	scanf("%d",&n);
	for(int i=0; i<n; i++){
		vector<int> v(n);
		for(int j=0; j<n; j++) scanf("%d",&v[j]);
		if(i % 2 == 1) inv ^= 1;
		insert(v);
	}
	if(inv) gyesu = mod - gyesu;
	gyesu = ipow(gyesu, mod - 2);
	for(int i=0; i<n; i++) gyesu = gyesu * basis[i][i] % mod;
	cout << gyesu % mod << endl; 
}
\end{minted}
\subsection{Simplex Algorithm}
\begin{minted}{cpp}
using T = long double;
const int N = 410, M = 30010;
const T eps = 1e-7;
int n, m;
int Left[M], Down[N];
// time complexity: exponential. fast $O(MN^2)$ in experiment. dependent on the modeling.
// Ax <= b, max c^T x. 최댓값: v, 답 추적: sol[i]. 1 based
T a[M][N], b[M], c[N], v, sol[N];
bool eq(T a, T b) { return fabs(a - b) < eps;  }
bool ls(T a, T b) { return a < b && !eq(a, b); }
void init(int p, int q) {
	n = p; m = q; v = 0;
	for(int i = 1; i <= m; i++){
		for(int j = 1; j <= n; j++) a[i][j]=0;
	}
	for(int i = 1; i <= m; i++) b[i]=0;
	for(int i = 1; i <= n; i++) c[i]=sol[i]=0;
}
void pivot(int x,int y) {
	swap(Left[x], Down[y]);
	T k = a[x][y]; a[x][y] = 1;
	vector<int> nz;
	for(int i = 1; i <= n; i++){
		a[x][i] /= k;
		if(!eq(a[x][i], 0)) nz.push_back(i);
	}
	b[x] /= k;

	for(int i = 1; i <= m; i++){
		if(i == x || eq(a[i][y], 0)) continue;
		k = a[i][y]; a[i][y] = 0;
		b[i] -= k*b[x];
		for(int j : nz) a[i][j] -= k*a[x][j];
	}
	if(eq(c[y], 0)) return;
	k = c[y]; c[y] = 0;
	v += k*b[x];
	for(int i : nz) c[i] -= k*a[x][i];
}
// 0: found solution, 1: no feasible solution, 2: unbounded
int solve() {
	for(int i = 1; i <= n; i++) Down[i] = i;
	for(int i = 1; i <= m; i++) Left[i] = n+i;
	while(1) { // Eliminating negative b[i]
		int x = 0, y = 0;
		for(int i = 1; i <= m; i++) if (ls(b[i], 0) && (x == 0 || b[i] < b[x])) x = i;
		if(x == 0) break;
		for(int i = 1; i <= n; i++) if (ls(a[x][i], 0) && (y == 0 || a[x][i] < a[x][y])) y = i;
		if(y == 0) return 1;
		pivot(x, y);
	}
	while(1) {
		int x = 0, y = 0;
		for(int i = 1; i <= n; i++)
			if (ls(0, c[i]) && (!y || c[i] > c[y])) y = i;
		if(y == 0) break;
		for(int i = 1; i <= m; i++)
			if (ls(0, a[i][y]) && (!x || b[i]/a[i][y] < b[x]/a[x][y])) x = i;
		if(x == 0) return 2;
		pivot(x, y);
	}
	for(int i = 1; i <= m; i++) if(Left[i] <= n) sol[Left[i]] = b[i];
	return 0;
}
\end{minted}
\subsection{Pentagonal Number Theorem for Partition Number Counting}
\begin{minted}{cpp}
vector<pair<int, int>> gp;
lint P[MAXN+1] = {};
gp.emplace_back(0, 0);
for(int i = 1; gp.back().second <= MAXN; i++) {
	gp.emplace_back(i % 2 ? 1 : -1, i * (3*i - 1) / 2);
	gp.emplace_back(i % 2 ? 1 : -1, i * (3*i + 1) / 2);
}
P[1] = 1;
for(int n = 2; n <= MAXN; n++) {
	for(auto it : gp) if(n >= it.second) P[n] += P[n - it.second] * it.first + MOD;
	P[n] %= MOD;
}
\end{minted}
\subsection{De Bruijn Sequence}
\begin{minted}{cpp}
// Create cyclic string of length k^n that contains every length n string as substring. alphabet = [0, k - 1]
int res[10000000]; // >= k^n
int aux[10000000]; // >= k*n
int de_bruijn(int k, int n) { // Returns size (k^n)
	if(k == 1) {
		res[0] = 0;
		return 1;
	}
	for(int i = 0; i < k * n; i++)
		aux[i] = 0;
	int sz = 0;
	function<void(int, int)> db = [&](int t, int p) {
		if(t > n) {
			if(n % p == 0)
				for(int i = 1; i <= p; i++)
					res[sz++] = aux[i];
		}
		else {
			aux[t] = aux[t - p];
			db(t + 1, p);
			for(int i = aux[t - p] + 1; i < k; i++) {
				aux[t] = i;
				db(t + 1, t);
			}
		}
	};
	db(1, 1);
	return sz;
}
\end{minted}
\subsection{Discrete Kth root}
\begin{minted}{cpp}
/* 
 * Solve x for x^P = A mod Q
 * (P, Q-1) = 1 -> P^-1 mod (Q-1) exists
 * x has solution iff A^((Q-1) / P) = 1 mod Q
 * PP | (Q-1) -> P < sqrt(Q), solve lgQ rounds of discrete log
 * else -> find a s.t. s | (Pa - 1) -> ans = A^a */
using LL = long long;
LL mul(LL x, LL y, LL mod){ return (__int128) x * y % mod; }
LL add(LL x, LL y, LL mod){ return (x + y) % mod; }
LL pw(LL x, LL y, LL mod){
	LL ret = 1, piv = x;
	while(y){
		if(y & 1) ret = mul(ret, piv, mod);
		piv = mul(piv, piv, mod);
		y >>= 1;
	}
	return ret % mod;
}
void gcd(LL a, LL b, LL &x, LL &y, LL &g){
	if (b == 0) {
		x = 1, y = 0, g = a;
		return;
	}
	LL tx, ty;
	gcd(b, a%b, tx, ty, g);
	x = ty; y = tx - ty * (a / b);
}
LL P, A, Q, g; // x^P = A mod Q
const int X = 1e5;
LL base, ae[X], aXe[X], iaXe[X];
unordered_map<LL, LL> ht;
#define FOR(i, c) for (int i = 0; i < (c); ++i)
#define REP(i, l, r) for (int i = (l); i <= (r); ++i)
void build(LL a) { // ord(a) = P < sqrt(Q)
	base = a;
	ht.clear();
	ae[0] = 1; ae[1] = a; aXe[0] = 1; aXe[1] = pw(a, X, Q);
	iaXe[0] = 1; iaXe[1] = pw(aXe[1], Q-2, Q);
	REP(i, 2, X-1) {
		ae[i] = mul(ae[i-1], ae[1], Q);
		aXe[i] = mul(aXe[i-1], aXe[1], Q);
		iaXe[i] = mul(iaXe[i-1], iaXe[1], Q);
	}
	FOR(i, X) ht[ae[i]] = i;
}

LL dis_log(LL x) {
	FOR(i, X) {
		LL iaXi = iaXe[i];
		LL rst = mul(x, iaXi, Q);
		if (ht.count(rst)) return i*X + ht[rst];
	}
}
LL main2() {
		cin >> P >> A >> Q;
	LL t = 0, s = Q-1;
	while (s % P == 0) {
		++t;
		s /= P;
	}
	if (A == 0) return 0;
	if (t == 0) {
		// a^{P^-1 mod phi(Q)}
		LL x, y, _;
		gcd(P, Q-1, x, y, _);
		if (x < 0) {
			x = (x % (Q-1) + Q-1) % (Q-1);
		}
		LL ans = pw(A, x, Q);
		if (pw(ans, P, Q) != A) while(1);
		return ans;
	}
	// A is not P-residue
	if (pw(A, (Q-1) / P, Q) != 1) return -1;
	for (g = 2; g < Q; ++g) {
		if (pw(g, (Q-1) / P, Q) != 1)
			break;
	}
	LL alpha = 0;
	{
		LL y, _;
		gcd(P, s, alpha, y, _);
		if (alpha < 0) alpha = (alpha % (Q-1) + Q-1) % (Q-1);
	}
	if (t == 1) {
		LL ans = pw(A, alpha, Q);
		return ans;
	}
	LL a = pw(g, (Q-1) / P, Q);
	build(a);
	LL b = pw(A, add(mul(P%(Q-1), alpha, Q-1), Q-2, Q-1), Q);
	LL c = pw(g, s, Q);
	LL h = 1;
	LL e = (Q-1) / s / P; // r^{t-1}
	REP(i, 1, t-1) {
		e /= P;
		LL d = pw(b, e, Q);
		LL j = 0;
		if (d != 1) {
			j = -dis_log(d);
			if (j < 0) j = (j % (Q-1) + Q-1) % (Q-1);
		}
		b = mul(b, pw(c, mul(P%(Q-1), j, Q-1), Q), Q);
		h = mul(h, pw(c, j, Q), Q);
		c = pw(c, P, Q);
	}
	return mul(pw(A, alpha, Q), h, Q);
}
\end{minted}
\subsection{Miller-Rabin Test + Pollard Rho Factorization}
\begin{minted}{cpp}
namespace miller_rabin{
    lint mul(lint x, lint y, lint mod){ return (__int128) x * y % mod; }
	lint ipow(lint x, lint y, lint p){
		lint ret = 1, piv = x % p;
		while(y){
			if(y&1) ret = mul(ret, piv, p);
			piv = mul(piv, piv, p);
			y >>= 1;
		}
		return ret;
	}
	bool miller_rabin(lint x, lint a){
		if(x % a == 0) return 0;
		lint d = x - 1;
		while(1){
			lint tmp = ipow(a, d, x);
			if(d&1) return (tmp != 1 && tmp != x-1);
			else if(tmp == x-1) return 0;
			d >>= 1;
		}
	}
	bool isprime(lint x){
		for(auto &i : {2, 3, 5, 7, 11, 13, 17, 19, 23, 29, 31, 37}){
			if(x == i) return 1;
			if(x > 40 && miller_rabin(x, i)) return 0;
		}
		if(x <= 40) return 0;
		return 1;
	}
}

namespace pollard_rho{
	lint f(lint x, lint n, lint c){
		return (c + miller_rabin::mul(x, x, n)) % n;
	}
	void rec(lint n, vector<lint> &v){
		if(n == 1) return;
		if(n % 2 == 0){
			v.push_back(2);
			rec(n/2, v);
			return;
		}
		if(miller_rabin::isprime(n)){
			v.push_back(n);
			return;
		}
		lint a, b, c;
		while(1){
			a = rand() % (n-2) + 2;
			b = a;
			c = rand() % 20 + 1;
			do{
				a = f(a, n, c);
				b = f(f(b, n, c), n, c);
			}while(gcd(abs(a-b), n) == 1);
			if(a != b) break;
		}
		lint x = gcd(abs(a-b), n);
		rec(x, v);
		rec(n/x, v);
	}
	vector<lint> factorize(lint n){
		vector<lint> ret;
		rec(n, ret);
		sort(ret.begin(), ret.end());
		return ret;
	}
};
\end{minted}
\subsection {Highly Composite Numbers, Large Prime}
\begin{minted}{cpp}
	< 10^k          number     divisors   2 3 5 71113171923293137
	-------------------------------------------------------------
	1                    6            4   1 1
	2                   60           12   2 1 1
	3                  840           32   3 1 1 1
	4                 7560           64   3 3 1 1
	5                83160          128   3 3 1 1 1
	6               720720          240   4 2 1 1 1 1
	7              8648640          448   6 3 1 1 1 1
	8             73513440          768   5 3 1 1 1 1 1
	9            735134400         1344   6 3 2 1 1 1 1
	10          6983776800         2304   5 3 2 1 1 1 1 1
	11         97772875200         4032   6 3 2 2 1 1 1 1
	12        963761198400         6720   6 4 2 1 1 1 1 1 1
	13       9316358251200        10752   6 3 2 1 1 1 1 1 1 1
	14      97821761637600        17280   5 4 2 2 1 1 1 1 1 1
	15     866421317361600        26880   6 4 2 1 1 1 1 1 1 1 1
	16    8086598962041600        41472   8 3 2 2 1 1 1 1 1 1 1
	17   74801040398884800        64512   6 3 2 2 1 1 1 1 1 1 1 1
	18  897612484786617600       103680   8 4 2 2 1 1 1 1 1 1 1 1

	< 10^k    prime   # of prime          < 10^k            prime
	-------------------------------------------------------------
	1             7            4          10           9999999967
	2            97           25          11          99999999977
	3           997          168          12         999999999989
	4          9973         1229          13        9999999999971
	5         99991         9592          14       99999999999973
	6        999983        78498          15      999999999999989
	7       9999991       664579          16     9999999999999937
	8      99999989      5761455          17    99999999999999997
	9     999999937     50847534          18   999999999999999989
\end{minted}
NTT Prime:

$998244353 = 119 \times 2^{23} + 1$. Primitive root: $3$.

$985661441 = 235 \times 2^{22} + 1$. Primitive root: $3$.

$1012924417 = 483 \times 2^{21} + 1$. Primitive root: $5$.
\section{Miscellaneous}
\subsection {Mathematics}
\begin{itemize}
	\item \textbf{Tutte Matrix}. For a simple undirected graph $G$, Let $M$ be a matrix with entries $A_{i, j} = 0$ if $(i, j) \notin E$ and $A_{i, j} = -A_{j, i} = X$ if $(i, j) \in E$. $X$ could be any random value. If the determinants are non-zero, then a perfect matching exists, while other direction might not hold for very small probability.
	\item \textbf{Cayley's Formula}. Given a degree sequence $d_1, d_2 \cdots, d_n$ for each labeled vertices, there exists $\frac{(n-2)!}{(d_1 - 1)!(d_2 - 1)! \cdots (d_n - 1)!}$ spanning trees. Summing this for every possible degree sequence gives $n^{n-2}$. 
	\item \textbf{Kirchhoff's Theorem}. For a multigraph $G$ with no loops, define Laplacian matrix as $L = D - A$. $D$ is a diagonal matrix with $D_{i, i} = deg(i)$, and $A$ is an adjacency matrix. If you remove any row and column of $L$, the determinant gives a number of spanning trees.
	\item \textbf{Green's Theorem}. Let $C$ is positive, smooth, simple curve. $D$ is region bounded by $C$. \\
	{
    	\large
    	$\oint_C (L dx + M dy)  = \iint_D (\frac{\partial M}{\partial x} - \frac{\partial L}{\partial y}) $ 
    }\\
    
    To calculate area, $\dfrac{\partial M}{\partial x} - \dfrac{\partial L}{\partial y} = 1$, common selection is $M = \dfrac{1}{2} x, \ \ L = -\dfrac{1}{2} y$. \\
    
    Line integral of circle parametrized by $(x, y) = (x_C + r_C \cos \theta, \ y_C + r_C \sin \theta)$, when $\theta = t \theta_i + (1 - t) \theta_f$, is given as follows.:  $\dfrac{1}{2}( r_C (x_C (\sin \theta_f - \sin \theta_i) - y_C (\cos \theta_f - \cos \theta_i) ) + (\theta_f - \theta_i) r_C ^ 2)$.  \\
    
    Line integral of line parametrized by $(x, y) = t(x_1, y_1) + (1-t)(x_2, y_2)$ is given as follows.: $\dfrac{1}{2} (x_1 y_2 - x_2 y_1)$.
    \item \textbf{Burnside's lemma / Pólya enumeration theorem}. let $G$ and $H$ be groups of permutations of finite sets $X$ and $Y$. Let $c_m(g)$ denote the number of cycles of length $m$ in $g \in G$ when permuting $X$. The number of colorings of $X$ into $\left|Y\right|=n$ colors with exactly $r_i$ occurrences of the $i$-th color is the coefficient of $w_1^{r_1}\ldots w_n^{r_n}$ in the following polynomial: \\
    $P(w_1,\ldots ,w_n)=\frac{1}{\left|H\right|}\sum_{h\in H}\frac{1}{\left|G\right|}\sum_{g\in G}\prod_{m\ge 1}(\sum_{h^m(b)=b}(w_b^m))^{c_m(g)}$ \\

    When $H=\{I\}$ (No color permutation): \\
    $P(w_1,\ldots ,w_n)=\frac{1}{\left|G\right|}\sum_{g\in G}\prod_{m\ge 1}(w_1^m+\ldots +w_n^m)^{c_m(g)}$ \\

    Without the occurrence restriction: \\
    $P(1,\ldots,1)=\frac{1}{\left|G\right|}\sum_{g\in G}n^{c(g)}$ \\
    where $c(g)$ could also be interpreted as the number of elements in $X$ that are fixed up to $g$.
    \item \textbf{Pick's Theorem}. $A=i+\frac{b}{2}-1$, where: $P$ is a simple polygon whose vertices are grid points, $A$ is area of $P$, $i$ is \# of grid points in the interior of $P$, and $b$ is \# of grid points on the boundary of $P$. \\
    If $h$ is \# of holes of $P$ ($h+1$ simple closed curves in total), $A=i+\frac{b}{2}+h-1$.
\begin{minted}{cpp}
// number of (x, y) : (0 <= x < n && 0 < y <= k/d x + b/d)
// argument should be positive
ll count_solve(ll n, ll k, ll b, ll d) {
	if (k == 0) {
		return (b / d) * n;
	}
	if (k >= d || b >= d) {
		return ((k / d) * (n - 1) + 2 * (b / d)) * n / 2 + count_solve(n, k % d, b % d, d);
	}
	return count_solve((k * n + b) / d, d, (k * n + b) % d, k);
}
\end{minted}
    \item \textbf{Xudyh Sieve}.
    $F(n)=\sum_{d\vert n}f(d)$ \\
    $S(n)=\sum_{i\leq n}f(i)=\sum_{i\leq n}F(i)-\sum_{d=2}^n S\left(\left\lfloor \frac{n}{d}\right\rfloor\right)$ \\
    \text{Preprocess $S(1)$ to $S(M)$\hspace{0.5cm}(Set $M=n^{\frac{2}{3}}$ for complexity)} \\
    $S(n)=\sum f(i) = \sum_{i\leq n}\left[F(i)-\sum_{j|i,j\neq i}f(j)\right]=\sum F(i) - \sum_{i/j=d=2}^n\sum_{dj\leq n}f(j)$ \\
    $S(n)=\sum if(i) = \sum_{i\leq n}i\left[F(i)-\sum_{j|i,j\neq i}f(j)\right]=\sum iF(i) - \sum_{i/j=d=2}^n\sum_{dj\leq n}djf(j)$ \\
    $\sum_{d\vert n}\varphi (d)=n\hspace{0.5cm}\sum_{d\vert n}\mu (d)=\text{if } (n > 1) \text{ then } 0 \text{ else } 1\hspace{0.5cm}\sum_{d\vert n}(\mu (\frac{n}{d})\sum_{e\vert d}f(e))=f(n)$ \\
\end{itemize}
\subsection {Popular Optimization Technique}
\begin{itemize}
	\item CHT. DnC optimization. Mo's \sout{algorithm} trick (on tree). IOI 2016 Aliens trick. IOI 2009 Regions trick.
	\item Knuth's $O(n^2)$ Optimal BST : minimize $D_{i, j} = Min_{i \leq k < j}(D_{i, k} + D_{k+1, j}) + C_{i, j}$. Quadrangle Inequality : $C_{a, c} + C_{b, d} \leq C_{a, d} + C_{b, c}$, $C_{b, c} \leq C_{a, d}$. Now monotonicity holds.
	\item Sqrt batch processing - Save queries in buffer, and update in every sqrt steps (cf : IOI 2011 Elephant. hyea calls it "ainta technique")
	\item Dynamic insertion in static set (Make $O(\log n)$ copy. Merge like binomial heap.)
	\item Offline insertion / deletion in insert-only set (Pair insertion-deletion operation, and regard it as range query)
	\item Atcoder Median Pyramid : Reduce the input to binary, and solve the easier problem.
	\item LP Duality. max $c^Tx$ sjt to $Ax \leq b$. Dual problem is min $b^Tx$ sjt to $A^Tx \geq c$. By strong duality, min max value coincides.
\end{itemize}
\subsection {Fast LL Division / Modulo}
\begin{minted}{cpp}
inline void fasterLLDivMod(unsigned long long x, unsigned y, unsigned &out_d, unsigned &out_m) {
	unsigned xh = (unsigned)(x >> 32), xl = (unsigned)x, d, m;
#ifdef __GNUC__
	asm(
		"divl %4; \n\t"
		: "=a" (d), "=d" (m)
		: "d" (xh), "a" (xl), "r" (y)
	);
#else
	__asm {
		mov edx, dword ptr[xh];
		mov eax, dword ptr[xl];
		div dword ptr[y];
		mov dword ptr[d], eax;
		mov dword ptr[m], edx;
	};
#endif
	out_d = d; out_m = m;
}
//x < 2^32 * MOD !
inline unsigned Mod(unsigned long long x){
	unsigned y = mod;
	unsigned dummy, r;
	fasterLLDivMod(x, y, dummy, r);
	return r;
}
\end{minted}
\subsection {Bit Twiddling Hack}
\begin{minted}{cpp}
int __builtin_clz(int x);// number of leading zero
int __builtin_ctz(int x);// number of trailing zero
int __builtin_clzll(long long x);// number of leading zero
int __builtin_ctzll(long long x);// number of trailing zero
int __builtin_popcount(int x);// number of 1-bits in x
int __builtin_popcountll(long long x);// number of 1-bits in x

lsb(n): (n & -n); // last bit (smallest)
floor(log2(n)): 31 - __builtin_clz(n | 1);
floor(log2(n)): 63 - __builtin_clzll(n | 1);

// compute next perm. ex) 00111, 01011, 01101, 01110, 10011, 10101..
long long next_perm(long long v){
	long long t = v | (v-1);
	return (t + 1) | (((~t & -~t) - 1) >> (__builtin_ctz(v) + 1));
}
\end{minted}
\subsection {Fast Integer IO}
\begin{minted}{cpp}
static char buf[1 << 19]; // size : any number geq than 1024
static int idx = 0;
static int bytes = 0;
static inline int _read() {
	if (!bytes || idx == bytes) {
		bytes = (int)fread(buf, sizeof(buf[0]), sizeof(buf), stdin);
		idx = 0;
	}
	return buf[idx++];
}
static inline int _readInt() {
	int x = 0, s = 1;
	int c = _read();
	while (c <= 32) c = _read();
	if (c == '-') s = -1, c = _read();
	while (c > 32) x = 10 * x + (c - '0'), c = _read();
	if (s < 0) x = -x;
	return x;
}
\end{minted}
\subsection {OSRank in g++}
\begin{minted}{cpp}
#include <ext/pb_ds/assoc_container.hpp>
#include <ext/pb_ds/tree_policy.hpp>
using namespace __gnu_pbds;

typedef
tree<int, null_type, less<int>, rb_tree_tag, tree_order_statistics_node_update> ordered_set;

ordered_set X;
X.insert(1); X.insert(2); X.insert(4); X.insert(8); X.insert(16);

cout<<*X.find_by_order(1)<<endl; // 2
cout<<*X.find_by_order(2)<<endl; // 4
cout<<*X.find_by_order(4)<<endl; // 16
cout<<(end(X)==X.find_by_order(6))<<endl; // true

cout<<X.order_of_key(-5)<<endl;  // 0
cout<<X.order_of_key(1)<<endl;   // 0
cout<<X.order_of_key(3)<<endl;   // 2
cout<<X.order_of_key(4)<<endl;   // 2
cout<<X.order_of_key(400)<<endl; // 5
\end{minted}
\subsection {Nasty Stack Hacks}
\begin{minted}{cpp}
// 64bit ver.
int main2(){ return 0; }
int main(){
	size_t  sz = 1<<29;  // 512MB
	void* newstack = malloc(sz);
	void* sp_dest = newstack + sz - sizeof(void*);
	asm  __volatile__("movq %0, %%rax\n\t"
	"movq %%rsp , (%%rax)\n\t"
	"movq %0, %%rsp\n\t": : "r"(sp_dest): );
	main2();
	asm  __volatile__("pop %rsp\n\t");
	return  0;
}
\end{minted}
\subsection {C++ / Environment Overview}
\begin{minted}{cpp}
// vimrc : set nu sc ci si ai sw=4 ts=4 bs=2  mouse=a syntax on

// compile : g++ -o PROB PROB.cpp -std=c++11 -Wall -O2
// options : -fsanitize=address -Wfatal-errors

struct StupidGCCCantEvenCompileThisSimpleCode{
	pair<int, int> array[1000000];
}; // https://gcc.gnu.org/bugzilla/show_bug.cgi?id=68203

// how to use rand (in 2018)
mt19937 rng(0x14004);
int randint(int lb, int ub){ return uniform_int_distribution<int>(lb, ub)(rng); }

// comparator overload
auto cmp = [](seg a, seg b){ return a.func() < b.func(); };
set<seg, decltype(cmp)> s(cmp);
map<seg, int, decltype(cmp)> mp(cmp);
priority_queue<seg, vector<seg>, decltype(cmp)> pq(cmp); // max heap

// hash func overload
struct point{
    int x, y;
    bool operator==(const point &p)const{ return x == p.x && y == p.y; }
};
struct hasher {
    size_t operator()(const point &p)const{ return p.x * 2 + p.y * 3; }
};
unordered_map<point, int, hasher> hsh;
\end{minted}
\subsection {Credits}
\begin{itemize}[noitemsep,nolistsep]
	\item cki86201, zigui, PavelKunyavskiy
	\item \url{https://gist.github.com/msg555/4963794}
	\item \url{https://github.com/niklasb/contest-algos/blob/master/convex_hull/dynamic.cpp}
	\item \url{https://github.com/jaehyunp/stanfordacm}
	\item \url{https://github.com/tzupengwang/PECaveros/blob/master/codebook/graph/BorrowedGeneralWeightedMatching.cpp}
	\item \url{https://github.com/tzupengwang/PECaveros/blob/master/codebook/math/DiscreteKthsqrt.cpp}
	\item \url{http://www-math.mit.edu/~etingof/groups.pdf}
\end{itemize}
\end{document}
